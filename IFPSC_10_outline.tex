\documentclass[11pt,a4paper]{article}
\usepackage{graphicx}
% uncomment according to your operating system:
% ------------------------------------------------
\usepackage[latin1]{inputenc}    %% european characters can be used (Windows, old Linux)
%\usepackage[utf8]{inputenc}     %% european characters can be used (Linux)
%\usepackage[applemac]{inputenc} %% european characters can be used (Mac OS)
% ------------------------------------------------
\usepackage{authblk}
\usepackage[superscript]{cite}
\usepackage[document]{ragged2e}
\usepackage[T1]{fontenc}   %% get hyphenation and accented letters right
\usepackage{mathptmx}      %% use fitting times fonts also in formulas
% do not change these lines:
\pagestyle{empty}                %% no page numbers!
\usepackage[left=35mm, right=35mm, top=15mm, bottom=20mm, noheadfoot]{geometry}
%% please don't change geometry settings!

\usepackage{fullpage}
\usepackage{amsfonts}
\usepackage{graphicx}
\usepackage{float}
\usepackage{amsmath}
\usepackage{chemfig}
\usepackage{indentfirst}
\usepackage{longtable}
\usepackage{array}
\usepackage{cellspace}
\usepackage{palatino}
%\usepackage{breqn}
\usepackage{amssymb}
\usepackage{verbatim}
\usepackage[colorlinks=true,citecolor=blue,linkcolor=blue]{hyperref}
\usepackage{siunitx}
\usepackage{xr}

% italicized boldface for math (e.g. vectors)
\newcommand{\bfv}[1]{{\mbox{\boldmath{$#1$}}}}
% non-italicized boldface for math (e.g. matrices)
\newcommand{\bfm}[1]{{\bf #1}}          

%\newcommand{\bfm}[1]{{\mbox{\boldmath{$#1$}}}}
%\newcommand{\bfm}[1]{{\bf #1}}
\newcommand{\expect}[1]{\left \langle #1 \right \rangle} % <.> for denoting expectations over realizations of an experiment or thermal averages

\newcommand{\var}[1]{{\mathrm var}{(#1)}}
\newcommand{\x}{\bfv{x}}
\newcommand{\y}{\bfv{y}}
\newcommand{\f}{\bfv{f}}

\newcommand{\hatf}{\hat{f}}

\newcommand{\bTheta}{\bfm{\Theta}}
\newcommand{\btheta}{\bfm{\theta}}
\newcommand{\bhatf}{\bfm{\hat{f}}}
\newcommand{\Cov}[1] {\mathrm{cov}\left( #1 \right)}
\newcommand{\T}{\mathrm{T}}                                % T used in matrix transpose

\newcommand\blfootnote[1]{%
	\begingroup
	\renewcommand\thefootnote{}\footnote{#1}%
	\addtocounter{footnote}{-1}%
	\endgroup
}


% begin the document
\begin{document}
	\thispagestyle{empty}
	%make title bold and 14 pt font (Latex default is non-bold, 16 pt)
	\title{\Large \textbf{The role of force field uncertainty in the prediction of high pressure viscosities.}}
	\author[1]{\large {\underline{Richard Messerly}}}%%[12 pt regular, presenting speaker underlined]
	
	\affil[1]{\textit{Thermodynamics Research Center (TRC), National Institute of Standards and Technology (NIST),
			Boulder, Colorado, 80305, USA}}
	
	\date{} % <--- leave date empty
	\maketitle\thispagestyle{empty} %% <-- you need this for the first page
	\begin{center}
		\title{\textbf{ABSTRACT}}\centering{}
	\end{center}
	\justify
	
\section*{Key points}

Force field parameter uncertainty is negligible compared to Green-Kubo uncertainty
We quantify both Mie and torsional uncertainties

\section*{Outline}

\section{Introduction}

\begin{enumerate}
	\item Introduce the industrial fluid properties simulation challenge
	\item Discuss the details of the 10th challenge
	\item Explain why this challenge is important/interesting:
	\begin{enumerate}
		\item Viscosity is an important property for designing chemical systems
		\item Viscosity data typically do not cover the entire range of $P \rho T$ of interest
		\item Prediction methods are typically quite poor for viscosity
		\item Molecular simulation is an attractive alternative, but two main challenges
		\begin{enumerate}
			\item Difficulty of obtaining reproducible results from simulation
			\item Unreliable force fields
		\end{enumerate}
	\end{enumerate}
	\item We performed a systematic investigation of several united-atom force fields and determined Potoff to be the most reliable
	\item Although Potoff over predicts viscosity and pressure with respect to density, it is quite reliable at predicting viscosity with respect to pressure
	\item The uncertainty in force field parameters is key for rigorously quantifying the uncertainty
\end{enumerate}

\section{Methods}

\subsection{Simulation set-up}

\begin{enumerate}
	\item NPT performed for each replicate such that a distribution of box sizes is obtained
	\item Depending on the system, a simulation of 1, 2, 4, 8, 16, or 32 ns was used for the production stage
	\item Details are in supporting information
\end{enumerate}

\subsection{Data analysis}

Refer to Zhang, Best Practices, and Special Issue manuscript

\begin{enumerate}
	\item Use 40\% sigma for cut-off
	\item Fit sigma to power model
	\item Fit viscosity to double exponential
	\item Bootstrap uncertainties by resampling replicate simulations
	\item 12 time origins
\end{enumerate}

\subsection{Force fields}

Copy the majority of this section from a previous publication

\begin{enumerate}
	\item Potoff force field proved to be most reliable in previous study
	\item United-atom, Mie 16-6
	\item AUA4m considered modifying torsional barriers for CH$_2$-CH$_2$ by 15~\% and 40~\% for internal and terminal torsions, respectively.
	\item Include uncertainty in $\epsilon$, $\sigma$, and $U^{\rm tors}$
	\item Plots of MCMC samples and maybe the Mie potentials and torsional barriers explicitly
\end{enumerate}

\section{Results}

\begin{enumerate}
	\item Potoff results
	\item Mie uncertainties are negligible
	\item Torsional uncertainties are negligible
	\item Mie and torsional uncertainties are negligible
	\item Report estimated pressure coefficient
\end{enumerate}

Figures:

\begin{enumerate}
	\item Viscosity vs pressure
	\item Bootstrap uncertainties
	\item Validation of bootstrap approach for ethane?
	\item Convergence of Green-Kubo plateau for 1000 MPa
\end{enumerate}

\section{Discussion/Limitations}

\begin{enumerate}
	\item Discussion
	\begin{enumerate}
		\item Branched alkanes are not as accurate, perhaps assumption of transferability or torsional parameters
		\item Force field parameter uncertainties are not as significant as often perceived
		\item Fixed bond lengths, 1.4 nm cut-off with tail corrections, 400 molecules, were all validated previously
	\end{enumerate}
    \item Limitations
    \begin{enumerate}
    	\item Largest viscosity simulations are slow to converge and unclear if simulations are sufficiently long
    	\item Tail-corrections could impact dynamics
    	\item Finite size effects
    \end{enumerate}
\end{enumerate}

\section{Conclusions}

\section{Acknowledgments}

\section{Supporting Information}

\subsection{Gromacs input files}

\begin{enumerate}
	\item Include all the .gro files
	\item Include all the .top file templates
	\item Include .mdp files
	\item Or we can just include an example and then refer them to the GitHub website
\end{enumerate}

\subsection{Tabulated values}

\subsection{Green-Kubo plateau plots}

\subsection{Bayesian bootstrap analysis}

Example analysis, i.e., bootstrap distribution, replicates

\end{document}
