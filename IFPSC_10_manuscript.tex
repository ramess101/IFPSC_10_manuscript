%% 
%% Copyright 2007, 2008, 2009 Elsevier Ltd
%% 
%% This file is part of the 'Elsarticle Bundle'.
%% ---------------------------------------------
%% 
%% It may be distributed under the conditions of the LaTeX Project Public
%% License, either version 1.2 of this license or (at your option) any
%% later version.  The latest version of this license is in
%%    http://www.latex-project.org/lppl.txt
%% and version 1.2 or later is part of all distributions of LaTeX
%% version 1999/12/01 or later.
%% 
%% The list of all files belonging to the 'Elsarticle Bundle' is
%% given in the file `manifest.t\textbf{xt}'.
%% 

%% Template article for Elsevier's document class `elsarticle'
%% with numbered style bibliographic references
%% SP 2008/03/01

\documentclass[preprint,review,12pt]{elsarticle}

%% Use the option review to obtain double line spacing
%%\documentclass[authoryear,preprint,review,12pt]{elsarticle}

%% Use the options 1p,twocolumn; 3p; 3p,twocolumn; 5p; or 5p,twocolumn
%% for a journal layout:
%% \documentclass[final,1p,times]{elsarticle}
%% \documentclass[final,1p,times,twocolumn]{elsarticle}
%% \documentclass[final,3p,times]{elsarticle}
%% \documentclass[final,3p,times,twocolumn]{elsarticle}
%% \documentclass[final,5p,times]{elsarticle}
%% \documentclass[final,5p,times,twocolumn]{elsarticle}

%% For including figures, graphicx.sty has been loaded in
%% elsarticle.cls. If you prefer to use the old commands
%% please give \usepackage{epsfig}

%% The amssymb package provides various useful mathematical symbols
\usepackage{amssymb}
%% The amsthm package provides extended theorem environments
%% \usepackage{amsthm}

%% The lineno packages adds line numbers. Start line numbering with
%% \begin{linenumbers}, end it with \end{linenumbers}. Or switch it on
%% for the whole article with \linenumbers.
%% \usepackage{lineno}

\usepackage{fullpage}
\usepackage{amsfonts}
\usepackage{graphicx}
\usepackage{amsmath}
\usepackage{indentfirst}
\usepackage[version=3]{mhchem} % Formula subscripts using \ce{}
\usepackage[T1]{fontenc}       % Use modern font encodings

\usepackage{float}
\usepackage{chemfig}
\usepackage{longtable}
\usepackage{array}
\usepackage{cellspace}
\usepackage{palatino}
%\usepackage{breqn}
\usepackage{amssymb}
\usepackage{verbatim}
\usepackage[colorlinks=true,citecolor=blue,linkcolor=blue]{hyperref}
\usepackage{siunitx}
\usepackage{xr}

%%% Old arguments
%\usepackage{graphicx}
%% uncomment according to your operating system:
%% ------------------------------------------------
%\usepackage[latin1]{inputenc}    %% european characters can be used (Windows, old Linux)
%%\usepackage[utf8]{inputenc}     %% european characters can be used (Linux)
%%\usepackage[applemac]{inputenc} %% european characters can be used (Mac OS)
%% ------------------------------------------------
%\usepackage{authblk}
%\usepackage[superscript]{cite}
%\usepackage[document]{ragged2e}
%\usepackage[T1]{fontenc}   %% get hyphenation and accented letters right
%\usepackage{mathptmx}      %% use fitting times fonts also in formulas
%% do not change these lines:
%\pagestyle{empty}                %% no page numbers!
%\usepackage[left=35mm, right=35mm, top=15mm, bottom=20mm, noheadfoot]{geometry}
%%% please don't change geometry settings!
%
%\usepackage{fullpage}
%\usepackage{amsfonts}
%\usepackage{graphicx}
%\usepackage{float}
%\usepackage{amsmath}
%\usepackage{chemfig}
%\usepackage{indentfirst}
%\usepackage{longtable}
%\usepackage{array}
%\usepackage{cellspace}
%\usepackage{palatino}
%%\usepackage{breqn}
%\usepackage{amssymb}
%\usepackage{verbatim}
%\usepackage[colorlinks=true,citecolor=blue,linkcolor=blue]{hyperref}
%\usepackage{siunitx}
%\usepackage{xr}

%% italicized boldface for math (e.g. vectors)
%\newcommand{\bfv}[1]{{\mbox{\boldmath{$#1$}}}}
%% non-italicized boldface for math (e.g. matrices)
%\newcommand{\bfm}[1]{{\bf #1}}          
%
%%\newcommand{\bfm}[1]{{\mbox{\boldmath{$#1$}}}}
%%\newcommand{\bfm}[1]{{\bf #1}}
%\newcommand{\expect}[1]{\left \langle #1 \right \rangle} % <.> for denoting expectations over realizations of an experiment or thermal averages
%
%\newcommand{\var}[1]{{\mathrm var}{(#1)}}
%\newcommand{\x}{\bfv{x}}
%\newcommand{\y}{\bfv{y}}
%\newcommand{\f}{\bfv{f}}
%
%\newcommand{\hatf}{\hat{f}}
%
%\newcommand{\bTheta}{\bfm{\Theta}}
%\newcommand{\btheta}{\bfm{\theta}}
%\newcommand{\bhatf}{\bfm{\hat{f}}}
%\newcommand{\Cov}[1] {\mathrm{cov}\left( #1 \right)}
%\newcommand{\T}{\mathrm{T}}                                % T used in matrix transpose
%
%\newcommand\blfootnote[1]{%
%	\begingroup
%	\renewcommand\thefootnote{}\footnote{#1}%
%	\addtocounter{footnote}{-1}%
%	\endgroup
%}

\makeatletter
\newcommand*{\addFileDependency}[1]{% argument=file name and extension
	\typeout{(#1)}
	\@addtofilelist{#1}
	\IfFileExists{#1}{}{\typeout{No file #1.}}
}
\makeatother

\newcommand*{\myexternaldocument}[1]{%
	\externaldocument{#1}%
	\addFileDependency{#1.tex}%
	\addFileDependency{#1.aux}%
}

\myexternaldocument{IFPSC_10_supporting_information}

% The figures are in a figures/ subdirectory.
\graphicspath{{figures/}}

\journal{Fluid Phase Equilibria}

\begin{document}
	
	\begin{frontmatter}
		
		%% Title, authors and addresses
		
		%% use the tnoteref command within \title for footnotes;
		%% use the tnotetext command for theassociated footnote;
		%% use the fnref command within \author or \address for footnotes;
		%% use the fntext command for theassociated footnote;
		%% use the corref command within \author for corresponding author footnotes;
		%% use the cortext command for theassociated footnote;
		%% use the ead command for the email address,
		%% and the form \ead[url] for the home page:
		%% \title{Title\tnoteref{label1}}
		%% \tnotetext[label1]{}
		%% \author{Name\corref{cor1}\fnref{label2}}
		%% \ead{email address}
		%% \ead[url]{home page}
		%% \fntext[label2]{}
		%% \cortext[cor1]{}
		%% \address{Address\fnref{label3}}
		%% \fntext[label3]{}
		
%		\title{The role of force field parameter uncertainty in the prediction of the pressure-viscosity coefficient}
%		\title{Mie 16-6 force field predicts super-Arrhenius viscosity behavior at high pressures for 2,2,4-trimethylhexane}
%		\title{Successful prediction of super-Arrhenius viscosity behavior at high pressures for 2,2,4-trimethylhexane}
%		\title{Probing the existence of super-Arrhenius pressure-viscosity coefficients with molecular simulation}
        \title{Mie 16-6 force field predicts viscosity with faster-than-exponential pressure dependence for 2,2,4-trimethylhexane}
				
		%% use optional labels to link authors explicitly to addresses:
		%% \author[label1,label2]{}
		%% \address[label1]{}
		%% \address[label2]{}
		
		\author{Richard A. Messerly}
		\ead{richard.messerly@nist.gov}
		\address{Thermodynamics Research Center, National Institute of Standards and Technology, Boulder, Colorado, 80305}
		
		\author{Michelle C. Anderson}
		\ead{michelle.anderson@nist.gov}
		\address{Thermodynamics Research Center, National Institute of Standards and Technology, Boulder, Colorado, 80305}
		
		\author{S. Mostafa Razavi}
		\address{Department of Chemical and Biomolecular Engineering, The University of Akron, Akron, Ohio, 44325-3906}
        \ead{sr87@zips.uakron.edu}
		
		\author{J. Richard Elliott}
		\address{Department of Chemical and Biomolecular Engineering, The University of Akron, Akron, Ohio, 44325-3906}
		\ead{elliot1@uakron.edu}
		
		%		
		%	\thispagestyle{empty}
		%	%make title bold and 14 pt font (Latex default is non-bold, 16 pt)
		%	\title{\Large \textbf{Transferability of Mie $\lambda$-6 force fields for predicting liquid shear viscosity at saturation and elevated pressures}}
		%
		%	\date{} % <--- leave date empty
		%	\maketitle\thispagestyle{empty} %% <-- you need this for the first page
		%	\begin{center}
		%		\title{\textbf{ABSTRACT}}\centering{}
		%	\end{center}
		%	\justify
		%	
		%	\author{Richard A. Messerly}
		%	\email{richard.messerly@nist.gov}
		%	\affiliation{Thermodynamics Research Center, National Institute of Standards and Technology, Boulder, Colorado, 80305}
		%	
		%	\author{Michael R. Shirts}
		%	\email{michael.shirts@colorado.edu}
		%	\affiliation{Department of Chemical and Biological Engineering, University of Colorado, Boulder, Colorado, 80309}
		%	
		%	\author{Andrei F. Kazakov}
		%	\email{andrei.kazakov@nist.gov}
		%	\affiliation{Thermodynamics Research Center, National Institute of Standards and Technology, Boulder, Colorado, 80305}
		
		\begin{abstract}
			
			In response to the 10$^{\rm th}$ Industrial Fluid Properties Simulation Challenge, we report viscosity $(\eta)$ estimates obtained with equilibrium molecular dynamics for 2,2,4-trimethylhexane at 293 K and over a range of pressures $(P)$ from 0.1 MPa to 1000 MPa. The Mie Potentials for Phase Equilibria (MiPPE) force field is utilized in this study, as a previous study demonstrated that it provides reliable estimates of $\eta$ with respect to $P$. Whereas most studies report only the uncertainties associated with random fluctuations in the simulation output and subsequent data analysis, we investigate the effect of uncertainties arising from the force field non-bonded and torsional parameters. The pressure-viscosity coefficient as a function of pressure is reported for several different empirical models, namely, McEwen-Paluch, Roelands, Roelands-Modified, and Barus. Although the uncertainties increase substantially with increasing pressure, cross-validation model selection provides quantitative evidence supporting faster-than-exponential, a.k.a. super-Arrhenius, behavior with an apparent inflection point in a log$_{10}(\eta)$-$P$ plot around 200 MPa. Near-quantitative agreement between simulation results and new experimental data is achieved for $P \le 600$ MPa, followed by significant over prediction at higher pressures.
			
			%Cross-validation model selection is employed to verify that a super-Arrhenius empirical equation is required to reproduce simulation results. 		
			
		\end{abstract}
		
		\begin{keyword}
			%% keywords here, in the form: keyword \sep keyword
			
			%% PACS codes here, in the form: \PACS code \sep code
			
			%% MSC codes here, in the form: \MSC code \sep code
			%% or \MSC[2008] code \sep code (2000 is the default)
			
			Pressure-viscosity coefficient \sep Industrial Fluid Properties Simulation Challenge \sep Uncertainty Quantification \sep Molecular Dynamics Simulation \sep Green-Kubo
			
		\end{keyword}
		
	\end{frontmatter}	
		
	\section{Introduction}
	
	The Industrial Fluid Properties Simulation Challenge (IFPSC) is an open international competition aimed at aligning the molecular simulation community, which is primarily academic, with the goals of industrial research. The present work is a submission to the 10$^{\rm th}$ Industrial Fluid Properties Simulation Challenge (IFPSC10)\footnote{http://fluidproperties.org/10th}, which challenges modelers to predict the viscosity $(\eta)$ of 2,2,4-trimethylhexane (224TMH) over a wide range of pressures $(P)$, specifically, from 0.1 MPa (atmospheric) to 1000 MPa, at a constant temperature $(T)$ of 293 K.
	
%	IFPSC is organized by several groups, namely, the Computational Molecular Science and Engineering Forum (CoMSEF) of the American Institute of Chemical Engineers (AIChE), the American Chemical Society (ACS), Army Research Lab, National Institute of Standards and Technology (NIST), The Dow Chemical Company, 3M, and United Technologies Research Center. 
    The practical application of IFPSC10 is elastohydrodynamic lubrication (EHL), where knowledge of the pressure-viscosity relationship is paramount. The challenge compound was chosen as an ideal lubricating oil candidate for which no published experimental viscosity data are available above ambient pressure. New experimental measurements have been performed by Scott Bair of Georgia Tech but were not made available prior to the challenge to avoid the experimental data biasing the simulations and to fairly test the molecular modeling approaches in a predictive manner. The sample purity is greater than 98~\% and the estimated experimental uncertainties for $\eta$, $T$, and $P$ are, respectively, 3~\%, 0.3 K, and the greater of 1 MPa and 0.4~\%.
	
	Classical film thickness formulas rely heavily on the so-called pressure-viscosity coefficient $(\alpha)$, which is essentially an Arrhenius-like activation parameter that is obtained from the slope of a log$_{10}(\eta)$-$P$ plot (see Section \ref{Pressure-viscosity coefficient}). However, faster-than-exponential, a.k.a. super-Arrhenius, dependence on pressure has been observed through experimental viscometry measurements for nearly a century \cite{Bair2016}. This super-Arrhenius trend is typically manifest by an inflection point in the log$_{10}(\eta)$-$P$ plot at high pressures. While this behavior is common in experimental measurements, we are not aware of any rheological molecular simulation studies that have addressed this topic, as most simply assume a linear relationship between log$_{10}(\eta)$ and $P$ \cite{Mundy1996,McCabe2001,Liu2015}. IFPSC10 is an ideal opportunity to demonstrate whether or not molecular simulation can provide evidence supporting or opposing the existence of super-Arrhenius behavior.
	
%	Most rheological molecular simulation studies assume an Arrhenius relationship and report the so-called pressure-viscosity coefficient which is the slope for an log$_{10}(\eta)-P$ plot \cite{Mundy1994,McCabe2001,Liu2015}. 
	
	In a previous study \cite{Postdoc_3}, we investigated four different united-atom (UA) Mie $\lambda$-6 (generalized Lennard-Jones, LJ) force fields, namely, the Transferable Potentials for Phase Equilibria (TraPPE-UA \cite{TraPPE,Martin1999,TraPPEUA2}), Transferable Anisotropic Mie (TAMie) \cite{TAMie,Weidler2016}, Mie Potentials for Phase Equilibria (MiPPE, referred to simply as ``Potoff'' in our previous study) \cite{Mie,Potoff_branched}, and fourth generation anisotropic-united-atom (AUA4) \cite{AUA4,Nieto2006,Nieto2008}. Molecular simulation results were compared with experimental data for saturated liquid viscosity $(\eta_{\rm liq}^{\rm sat})$ over a wide temperature range and compressed liquid viscosity $(\eta_{\rm liq}^{\rm comp})$ at 293 K from atmospheric pressure to 1000 MPa, i.e., the same temperature and pressure range as the challenge. The compounds in question were seven \textit{n}-alkanes ranging in length from ethane to \textit{n}-docosane and seven branched alkanes ranging in size from 2-methylpropane to 2,2,4-trimethylpentane (224TMP). The 224TMP results at high pressures are especially useful as this compound is a close analogue to the challenge compound and, in contrast with 224TMH, 224TMP has been well studied experimentally.
	
	While TraPPE and AUA4 (LJ 12-6 based potentials) under predict $\eta_{\rm liq}^{\rm sat}$ by 20~\% to 50~\% for all compounds studied, TAMie (Mie 14-6) and MiPPE (Mie 16-6) predict $\eta_{\rm liq}^{\rm sat}$ within 10~\% for most compounds \cite{Postdoc_3}. For $\eta_{\rm liq}^{\rm comp}$, TAMie is the most reliable at predicting the viscosity-density dependence, while MiPPE significantly over estimates $\eta_{\rm liq}^{\rm comp}$ with respect to density. However, since MiPPE also over estimates pressure at high densities \cite{Postdoc_2}, the pressure-viscosity trend for MiPPE is remarkably accurate even at pressures approaching 1000 MPa. In particular, the MiPPE force field predicts the pressure-viscosity trend for 224TMP to within 10~\%. For this reason, we implement the MiPPE (Mie 16-6) force field to predict $\eta$ and $\alpha$ for the challenge compound. We should note, however, that our previous study did not provide any definitive evidence that the MiPPE force field could predict a super-Arrhenius trend for any compound studied, including for 224TMP.
	 
	One of the entry guidelines for IFPSC is ``an analysis of the uncertainty in the calculated results.'' Traditionally, simulation uncertainties are limited to the random fluctuations of simulation output and/or the uncertainty related to data post-processing. This class of uncertainty is referred to as ``numerical uncertainty'' (also frequently referred to as ``statistical uncertainty'') \cite{Bay_Deriv,Bay_MD,Bay_UQ,Mess4}. Two other classes of uncertainty, namely, ``parameter uncertainty'' and ``functional form uncertainty'' (also referred to as ``model uncertainty'') are typically ignored in uncertainty quantification (UQ) due to the increased computational cost \cite{Bay_Deriv,Bay_MD,Bay_UQ,Mess4}. The latter refers to the uncertainty associated with the choice of force field functional form, while the former refers to the uncertainty in the force field parameters for a given force field functional form.
	
	%, although they are often larger than numerical uncertainty
	
	Quantifying the functional form uncertainty is an extremely difficult task, as it often requires testing numerous force field functional forms. For this reason, we focus on numerical and parameter uncertainties without addressing functional form uncertainties. Specifically, we apply bootstrap re-sampling \cite{Efron1979} and Bayesian inference Markov Chain Monte Carlo (MCMC) \cite{Postdoc_2,Bay_Deriv,Bay_MD,Bay_UQ} to quantify numerical and parameter uncertainties, respectively. The chosen functional form is the same as the MiPPE force field, namely, a united-atom, fixed bond length, harmonic angular potential, Fourier series torsional potential, and a Mie 16-6 non-bonded potential (see Section \ref{Force Field} for details). As viscosity is highly sensitive to the non-bonded \cite{Postdoc_3,Gordon2006} and torsional \cite{Nieto2006,Braga2012} potentials, we limit our parameter uncertainty investigation to the non-bonded and torsional parameters. 
	
	%We apply Bayesian inference Markov Chain Monte Carlo (MCMC) to quantify and propagate the parameter uncertainty.
	
	%quantify the uncertaint  a Bayesian inference Markov Chain Monte Carlo 
	
	%limit our parameter uncertainty investigation to the non-bonded and torsional parameters.
	
	The outline for the present work is the following. Section \ref{Methods} explains the force field, parameter uncertainties, simulation methodology, data analysis, and empirical models. Section \ref{Results} presents the simulation results, with an emphasis on uncertainty quantification. Section \ref{Discussion/Limitations} discusses some important observations and limitations. Section \ref{Conclusions} recaps the primary conclusions from this work.
	
%	The 10th Industrial Fluid Properties Simulation Challenge will test the capability of molecular dynamics simulation to provide the property of liquids most important to elastohydrodynamic lubrication (EHL), the pressure-viscosity relation.  The temperature dependence at elevated pressure could be the subject of a future challenge.
%	
%	A fundamental requirement of elastohydrodynamic lubrication (EHL) is a description of the viscosity of the liquid as a function of pressure [1].  The classical film thickness formulas all require a value for a property known as a pressure-viscosity coefficient; although the definition of this property is not always clear [2]. The shape of a traction (friction) curve has been the subject of much speculation for at least forty years [3]. The shape of a traction curve when plotted as friction coefficient or average shear stress versus the logarithm of sliding speed or slide-to-roll ratio depends strongly on the pressure dependence of viscosity at the Hertz pressure [4], more fragile liquids having a less steep logarithmic portion. Although the pressure dependence of viscosity is clearly essential to the field, until about ten years ago, experimentally measured values of this property were not a typical part of EHL analysis.  The reasons for the previous neglect may be debated, but the demand for this information is now growing.
%	
%	Molecular dynamics simulations have the promise of generating pressure-viscosity data for liquids which have not yet been synthesized but only if the accuracy of the method can be validated.  There has been a claim of success in predicting the pressure dependence of viscosity for squalane [5], although the temperature dependence is not accurately recovered in this example [6]. There has been extensive experimental work on squalane viscosity at elevated pressure [7,8] so that simulations have a known “target” value of viscosity. As of this time, there has been no success in recovering the super-Arrhenius pressure dependence that is important to friction [9].
%	
%	A Lubricant Viscosity Simulation Challenge is now proposed to assess the possibility of employing molecular dynamics simulations to predict the pressure dependence of viscosity in a simple hydrocarbon molecule.  This should be a material for which there is no presently published viscosity data, with the exception perhaps of viscosity at ambient pressure. It should, for simulation convenience, be composed of a minimum number of carbons.  Linear alkanes are excluded because they are not glass-formers and would be crystallized at EHL pressures.  The material should possess all of the pressure-viscosity trends of lubricating oil.  A candidate fulfilling these requirements is 2,2,4 Trimethylhexane.
%		
%	Prof. Scott Bair (Georgia Tech) has characterized the viscosity of 2,2,4 Trimethylhexane (>98\%), Aldrich product number 92470, lot BCBR3588V.  The viscometers were calibrated with di (2ethylhexyl) sebacate based on the correlation of Paredes et al. [10].  Estimated uncertainties are 3\% for viscosity, 0.3°C for temperature and the greater of 1MPa and 0.4\% for pressure.
%	
%	Entrants are challenged to predict the viscosity at pressures of 0.1, 25, 50, 100, 150, 250, 400, 500, 600, 700, 800, 900, and 1000 MPa, all at temperature of 20°C (293K).  Entries will be judged based on comparison to the benchmark data at those state conditions and to the pressure viscosity coefficient.
%	
%	\begin{enumerate}
%		\item Introduce the industrial fluid properties simulation challenge
%		\item Discuss the details of the 10th challenge
%		\item Explain why this challenge is important/interesting:
%		\begin{enumerate}
%			\item Viscosity is an important property for designing chemical systems
%			\item Viscosity data typically do not cover the entire range of $P \rho T$ of interest
%			\item Prediction methods are typically quite poor for viscosity
%			\item Molecular simulation is an attractive alternative, but two main challenges
%			\begin{enumerate}
%				\item Difficulty of obtaining reproducible results from simulation
%				\item Unreliable force fields
%			\end{enumerate}
%		\end{enumerate}
%		\item We performed a systematic investigation of several united-atom force fields and determined Potoff to be the most reliable
%		\item Although Potoff over predicts viscosity and pressure with respect to density, it is quite reliable at predicting viscosity with respect to pressure
%		\item The uncertainty in force field parameters is key for rigorously quantifying the uncertainty
%	\end{enumerate}  
        
	\section{Methods} \label{Methods}
	
		\subsection{Force field} \label{Force Field}
	
	We utilize the MiPPE force field, as it provides reliable estimates of the $\eta$-$P$ dependence for normal and branched alkanes that are similar to the challenge compound \cite{Postdoc_3}. In addition, we quantify the uncertainty in $\eta$ that is inherited from uncertainties in the non-bonded Mie 16-6 and torsional parameters. The parameter uncertainties are obtained using Bayesian inference Markov Chain Monte Carlo (MCMC). This UQ analysis is performed sequentially. First, we account for only the non-bonded uncertainties (referred to as MCMC-nb). Then, we include both the non-bonded and torsional uncertainties (MCMC-nb-tors). This sequential approach provides insight into which source of uncertainty has a greater impact on $\eta$.
	
%	 and are subsequently propagate these force field parameter uncertainties when predicting $\eta$.
	
	 %This is done sequentially, where we first include only the non-bonded uncertainties and then we include both the non-bonded and torsional uncertainties %(referred to as MCMC-nb and MCMC-nb-tors, respectively).  
	
	%, we utilize 
%	As we demonstrated in our previous force field comparison study \cite{Postdoc_3}, the Potoff Mie $\lambda$-6 force field provides reliable estimates of the $\eta$-$P$ dependence for normal and branched alkanes. For this reason, we utilize the Potoff Mie $\lambda$-6 force field to predict $\eta$. In addition, we quantify the uncertainties in the non-bonded and torsional parameters using Bayesian inference. We subsequently propagate these force field parameter uncertainties when predicting $\eta$.
	
%	As we demonstrated in our previous study \cite{Postdoc_3}, the Potoff Mie $\lambda$-6 force field provides the most reliable estimates of the $\eta$-$P$ dependence for normal and branched alkanes. In particular, this force field predicts the viscosity within 10~\% for 2,2,4-trimethylpentane up to 200 MPa and for propane up to 750 MPa. By contrast, the popular TraPPE-UA force field deviated by greater than 30~\%, with errors increasing with respect to pressure. Although the TAMie force field had similar accuracy to Potoff for \textit{n}-alkanes, the deviations were much greater for branched alkanes. For these reasons, we utilize the Potoff Mie $\lambda$-6 force field to predict $\eta$. In addition, we quantify the uncertainties in the non-bonded and torsional parameters using Bayesian inference. We subsequently propagate these force field parameter uncertainties when predicting $\eta$. 

    \subsubsection{MiPPE force field}
	
	The MiPPE (Mie $\lambda$-6) force field utilizes united-atom (UA) sites, where 2,2,4-trimethylhexane is represented with CH$_3$, CH$_2$, CH, and C UA sites (see Figure \ref{fig:224TMH}). Neighboring UA sites are separated by a fixed 0.154 nm bond length. Note that we observed in our previous study that $\eta$ increases by several percent when flexible bonds are employed instead of fixed bonds \cite{Postdoc_3}. Therefore, the choice of fixed bonds was not arbitrary and is a possible source of uncertainty for which we did not rigorously account. The primary reason we utilize fixed bonds is to reduce the fluctuations in the stress tensor and, thereby, decrease the numerical uncertainty of the viscosity estimate.
	
	\begin{figure}[htb!]
		\centering
		\includegraphics[width=3.2in]{UA_IFPSC10}
		\caption{United-atom representation of the challenge compound, 2,2,4-trimethylhexane. All bond lengths are 0.154 nm.} 
		\label{fig:224TMH}
	\end{figure}
	
	The angular contribution to internal energy is computed using a harmonic potential:
	\begin{equation}
	u^{\rm bend} = \frac{k_\theta}{2} \left(\theta-\theta_0\right)^2
	\end{equation}
	where $u^{\rm bend}$ is the bending energy, $\theta$ is the instantaneous bond angle, $\theta_0$ is the equilibrium bond angle (see Table \ref{tab:angles}), and $k_\theta$ is the harmonic force constant with $k_\theta/k_{\rm B} = 62500$ K/rad$^2$ for all bonding angles, where $k_{\rm B}$ is the Boltzmann constant. 
	
	\begin{table}[h!]
		\caption{MiPPE equilibrium bond angles $(\theta_0)$ \cite{Martin1999,Potoff_branched}. CH$_i$ and CH$_j$ represent CH$_3$, CH$_2$, CH, or C sites.} \label{tab:angles}
		\begin{center}
			\begin{tabular}{|c|c|}
				\hline
				Bending sites & $\theta_0$ (degrees) \\ \hline
				CH$_i$-CH$_2$-CH$_j$ & 114.0 \\ 
				CH$_i$-CH-CH$_j$ & 112.0 \\ 
				CH$_i$-C-CH$_j$ & 109.5 \\  
				\hline
			\end{tabular}
		\end{center} 
	\end{table}
	
	%	Dihedral torsional interactions are determined using a cosine series:
	%	\begin{equation}
	%	u^{\rm tors} = c_0 + c_1 [1+\cos{\phi}] + c_2 [1-\cos{2\phi}] + c_3 [1+\cos{3\phi}]
	%	\end{equation}
	%	where $u^{\rm tors}$ is the torsional energy, $\phi$ is the dihedral angle and $c_n$ are the Fourier constants listed in Table \ref{tab:torsions}. Note that $\phi$ is defined using a convention similar to IUPAC where $\phi = 180 \deg$ for the \textit{trans} conformation.
	
	Dihedral torsional interactions are determined using a modified cosine series:
	%	\begin{equation}
	%	u^{\rm tors} = c_0 + c_1 [1+\cos{\phi}] + c_2 [1-\cos{2\phi}] + c_3 [1+\cos{3\phi}] + A_{\rm s} \sin^2\left(\frac{3\phi}{2} + \ang{180}\right) = (c_0 - A_{\rm s}) + c_1 [1+\cos{\phi}] + c_2 [1-\cos{2\phi}] + \left(c_3 + \frac{A_{\rm s}}{2}\right) [1+\cos{3\phi}]
	%	\end{equation}
	\begin{multline} \label{eq:torsions}
	u^{\rm tors} = c_0 + c_1 [1+\cos{\phi}] + c_2 [1-\cos{2\phi}] + c_3 [1+\cos{3\phi}] + A_{\rm s} \sin^2\left[\frac{3}{2}(\phi + \ang{180})\right] \\ = (c_0 - A_{\rm s}) + c_1 [1+\cos{\phi}] + c_2 [1-\cos{2\phi}] + \left(c_3 + \frac{A_{\rm s}}{2}\right) [1+\cos{3\phi}]
	\end{multline}
	where $u^{\rm tors}$ is the torsional energy, $\phi$ is the dihedral angle, $c_n$ are the Fourier constants used in the MiPPE force field and listed in Table \ref{tab:torsions}, and $A_{\rm s} \sin^2\left[\frac{3}{2}(\phi + \ang{180})\right]$ is an additional term proposed by Nieto-Draghi et al. to shift the torsional barrier heights for normal and branched alkanes \cite{Nieto2006,Nieto2008}. We follow a convention similar to that of the International Union of Pure and Applied Chemistry (IUPAC) such that $\phi = \ang{180}$ for the \textit{trans} conformation \cite{Martin1999}, whereas Nieto-Draghi et al. define the \textit{trans} conformation as $\ang{0}$ or $\ang{360}$ \cite{Nieto2006,Nieto2008}, hence the $\phi+\ang{180}$ term in Equation \ref{eq:torsions}. As $\sin^2\left[\frac{3}{2}(\phi + \ang{180})\right]$ has a maximum value of 1 at $\ang{0}$, $\ang{120}$, $\ang{240}$, and $\ang{360}$, the torsional barriers located at these dihedral angles increase by $A_{\rm s}$. By contrast, this additional term does not shift $u^{\rm tors}$ for dihedral angles of $\ang{60}$, $\ang{180}$, and $\ang{300}$, which correspond to the equilibrium conformations of \textit{gauche}$^-$, \textit{trans}, and \textit{gauche}$^+$, respectively. 
		
		%Clearly, the non-shifted Potoff torsional potential is obtained only when $A_{\rm s} = 0$. The actual reason we include this additional torsion term is to provide a simple method for quantifying the uncertainty in the torsional potential (see Section \ref{sec:parameter_uncertainty}). 
		
	\begin{table}[h!]
		\caption{MiPPE fourier constants $(c_n/k_{\rm B})$ and shifting parameter $(A_{\rm s}/k_{\rm B})$ in units of K \cite{Martin1999,Potoff_branched}. CH$_i$ and CH$_j$ represent CH$_3$, CH$_2$, CH, or C sites.} \label{tab:torsions}
		\begin{center}
			\begin{tabular}{|c|c|c|c|c|c|}
				\hline
				Torsion sites & $c_0/k_{\rm B}$ & $c_1/k_{\rm B}$ & $c_2/k_{\rm B}$ & $c_3/k_{\rm B}$ & $A_{\rm s}/k_{\rm B}$ \\ \hline
				CH$_i$-CH$_2$-CH-CH$_j$ & -251.06 & 428.73 & -111.85 & 441.27 & 0.0 \\
				CH$_i$-CH$_2$-C-CH$_j$ & 0.0 & 0.0 & 0.0 & 461.29 & 0.0 \\
				\hline
			\end{tabular}
		\end{center} 
	\end{table}

	Note that by expressing the $\sin^2\left[\frac{3}{2}(\phi + \ang{180})\right]$ term as $\cos{3\phi}$, this additional term does not actually modify the torsional functional form. Instead, it only affects $c_0$ and $c_3$ according to $c_0^{'} = c_0 - A_{\rm s}$ and $c_3^{'} = c_3 + \frac{A_{\rm s}}{2}$, where $c_0^{'}$ and $c_3^{'}$ are the shifted Fourier constants. Clearly, the non-shifted MiPPE torsional potential is obtained only when $A_{\rm s} = 0$. The real reason we include this additional torsion term is to provide a simple method for quantifying the torsional parameter uncertainties (see Section \ref{sec:parameter_uncertainty}). 
	
	Non-bonded interactions between sites located in two different molecules or separated by more than three bonds within the same molecule are calculated using a Mie $\lambda$-6 potential (of which the traditional Lennard-Jones, LJ, 12-6 is a subclass) \cite{Herdes2015}:
	\begin{equation} \label{eq:Mie}
	u^{\rm vdw}(\epsilon,\sigma,\lambda;r) = \left(\frac{\lambda}{\lambda - 6}\right)\left(\frac{\lambda}{6}\right)^{\frac{6}{\lambda - 6}} \epsilon \left[\left(\frac{\sigma}{r}\right)^{\lambda} - \left(\frac{\sigma}{r}\right)^6\right]
	\end{equation} 
	where $u^{\rm vdw}$ is the van der Waals interaction, $\sigma$ is the distance $(r)$ where $u^{\rm vdw} = 0$, $-\epsilon$ is the energy of the potential at the minimum $\left(\text{i.e., }u^{\rm vdw} = -\epsilon \text{ and } \frac{\partial u^{\rm vdw}}{\partial r} = 0 \text{ for } r=r_{\rm min} \right)$, and $\lambda$ is the repulsive exponent. 
	
	The MiPPE non-bonded parameters are provided in Table \ref{tab:nonbonded params}. Note that MiPPE provides a ``generalized'' and ``short/long'' (S/L) CH and C parameter set. The ``generalized'' CH and C parameter set is an attempt at a completely transferable force field, while the ``short'' and ``long'' parameters are implemented when the number of carbons in the backbone is $\le 4$ and $> 4$, respectively. As the longest continuous carbon backbone for 2,2,4-trimethylhexane consists of six carbons, the MiPPE results presented in this study are obtained with the ``long'' parameters.
	
	\begin{table}[h!]
		\caption{MiPPE non-bonded (Mie $\lambda$-6) parameters \cite{Mie,Potoff_branched}. The CH and C values are the ``long'' MiPPE parameters.} \label{tab:nonbonded params}
		\begin{center}
			\begin{tabular}{|c|c|c|c|}
				\hline
				\multicolumn{1}{|c}{} & \multicolumn{3}{|c|}{MiPPE}  \\ \hline
				United-atom & $\epsilon/k_{\rm B}$ (K) & $\sigma$ (nm) & $\lambda$ \\ \hline
				CH$_3$ & 121.25 & 0.3783 & 16  \\ 
				CH$_2$ & 61 & 0.399 & 16 \\ 
				CH & 14 & 0.47 & 16\\
				C & 1.2 & 0.62 & 16\\
				\hline
			\end{tabular}
		\end{center} 

%	\footnotesize
%	    $^*$Note in Reference \citenum{Potoff_branched} that a small discrepancy exists between the generalized C parameters in Table 1 and the optimal region depicted in Figure 1. The generalized C parameters reported here were obtained from private communication of tabulated scoring function values and are, thus, slightly different than those reported in Reference \citenum{Potoff_branched}.
	\end{table}

\normalsize
	
	Non-bonded parameters between two different site types (i.e., cross-interactions) are determined using Lorentz-Berthelot combining rules for $\epsilon$ and $\sigma$ \cite{Allen2017}:
	% and an arithmetic mean for the repulsive exponent $\lambda$ (as recommended in Reference \citenum{Mie}):
	\begin{equation} \label{eq:Lorentz-Berthelot_eps}
	\epsilon_{ij} = \sqrt{\epsilon_{ii} \epsilon_{jj}}
	\end{equation}
	\begin{equation} \label{eq:Lorentz-Berthelot_sig}
	\sigma_{ij} = \frac{\sigma_{ii} + \sigma_{jj}}{2}
	\end{equation}
%	\begin{equation} \label{eq:Lorentz-Berthelot_lam}
%	\lambda_{ij} = \frac{\lambda_{ii} + \lambda_{jj}}{2}
%	\end{equation}
	where the $ij$ subscript refers to cross-interactions and the subscripts $ii$ and $jj$ refer to same-site interactions. 
	
%	The MCMC Mie 16-6 parameters for CH$_3$ and CH$_2$ were reported previously \cite{Postdoc_2}. Reference BLANK assumed that the CH$_3$ parameters were transferable from ethane for longer \textit{n}-alkanes. The CH$_2$ parameters were obtained from propane, \textit{n}-butane, and \textit{n}-octane. The data included in the analysis were saturated liquid densities and saturated vapor pressures over a reduced temperature range of 0.45 to 0.85, as available in ThermoData Engine (TDE). 
%	

    \subsubsection{Parameter uncertainty} \label{sec:parameter_uncertainty}

	Nieto-Draghi et al. set $A_{\rm s}$ equal to 40\% and 15\% of the maximum dihedral barrier (with the \textit{cis} conformation) for the terminal and internal torsions, respectively \cite{Nieto2006,Nieto2008}. For example, this corresponds to $A_{\rm s}/k_{\rm B} \approx 1000$ K and $\approx 375$ K for the CH$_3$-CH$_2$-CH$_2$-CH$_2$ and CH$_2$-CH$_2$-CH$_2$-CH$_2$ torsional potentials, respectively. The primary reason Nieto-Draghi et al. increase the torsional barrier, i.e., $A_{\rm s} > 0$, is because AUA4 under predicts $\eta$ by approximately 20~\% to 40~\%. However, despite the relatively large increase in the torsional barriers, the modified force field (AUA4m) typically provides only marginal improvement of 5~\% to 15~\% compared to AUA4 (see Tables 4 and 5 of Reference \citenum{Nieto2008}). 
	
	% The primary reason why Nieto-Draghi et al. introduce this additional term is to increase the torsional barriers and, thereby, increase the viscosity obtained with the AUA4m force field, which systematically under predicts $\eta$ by around 30~\%. Despite the fairly large increase in torsional barrier,
	
	%	Reference \citenum{Nieto2006} set $A_{\rm s}$ equal to 40\% and 15\% of the maximum dihedral barrier for the CH$_3$-CH$_2$-CH$_2$-CH$_2$ and CH$_2$-CH$_2$-CH$_2$-CH$_2$ torsional potentials, respectively. This corresponds to $A_{\rm s}/k_{\rm B} \approx 1000$ K and $\approx 375$ K for the CH$_3$-CH$_2$-CH$_2$-CH$_2$ and CH$_2$-CH$_2$-CH$_2$-CH$_2$ torsional potentials, respectively. The primary reason why \citenum{Nieto2006} introduced this additional term was to increase the torsional barriers and, thereby, increase the viscosity obtained with the AUA4m force field. This methodology works fairly well for Lennard-Jones 12-6 force fields, which systematically under predict viscosity by greater than 30~\%. However, since the Potoff Mie 16-6 potential is already quite reliable for predicting viscosity, we would expect significant over prediction of viscosity if we coupled the Potoff Mie 16-6 potential with $A_{\rm s}/k_{\rm B} \gg 0 $ K.
	
	%$0.075 \times \max(u^{\rm tors}_{A_{\rm s}=0})$
	
	As the MiPPE potential is already quite reliable for predicting viscosity, we would expect significant over prediction of viscosity if we coupled the MiPPE (Mie 16-6) potential with $A_{\rm s}/k_{\rm B} \gg 0 $. Thus, unlike Nieto-Draghi et al., we do not propose that the torsional barriers must be increased unilaterally. Instead, we assume that $A_{\rm s}$ follows a skewed normal distribution with a mean value near zero and lower and upper 95~\% confidence intervals corresponding to -15~\% and +40~\% of the maximum barrier height for the non-shifted MiPPE torsional potential. The rationale for the $A_{\rm s}$ distribution is presented in Section \ref{SI:MCMC torsions} of Supplementary Material. The MCMC-nb-tors parameters $(A_{\rm s})$ sampled from this distribution are reported in Section \ref{SI:Tabulated_MCMC} of Supplementary Material.
	
	% Section \ref{Results}. 
	
	Figure \ref{fig:dihedral_uncertainty} compares the non-shifted MiPPE torsional potential, $\pm 40$~\% shift in barrier heights, and the MCMC-nb-tors potentials. The insets also depict the skewed distributions and the randomly sampled MCMC $A_{\rm s}$ sets. Note that the challenge compound consists of four CH$_i-$CH$_2-$CH$-$CH$_j$ torsions and three CH$_i-$CH$_2-$C$-$CH$_j$ torsions. Note that, unlike Nieto-Draghi et al., we make no distinction between internal and terminal torsions. 
	
	%	The actual reason we include this additional term, however, is to provide a simple method for quantifying the uncertainty in the torsional potential. Specifically, we assume that $A_{\rm s}$ follows a normal distribution with a mean value of zero and a standard deviation equal to $0.075 \times \max(u^{\rm tors}_{\rm Potoff})$. The standard deviation is assigned such that the 95\% confidence interval is equal to 15\% the maximum barrier height for the Potoff torsional potential. We use a normal distribution such that the uncertainty in the dihedral barriers is symmetric, i.e., unlike \citenum{Nieto2006} we do not assume that the dihedral barriers must be increased unilaterally. Figure \ref{fig:dihedral_uncertainty} compares the Potoff and MCMC torsional potentials. The insets also depict the normal distribution and the randomly sampled MCMC $A_{\rm s}$ sets. Note that the challenge compound consists of four CH$_i-$CH$_2-$CH$-$CH$_j$ torsions and three CH$_i-$CH$_2-$C$-$CH$_j$ torsions.
	
	%Note that \citenum{Nieto2006} proposed a 15\% shift in the dihedral barriers for the CH$_i$-CH$_2$-CH$_2$-CH$_j$ potential.
	
	\begin{figure}[htb!]
		\centering
		\includegraphics[width=6.4in]{MCMC_torsions.pdf}
		\caption{Comparison of MiPPE (black solid lines), $\pm 40$~\% (green dotted lines), and MCMC-nb-tors (red dashed lines) torsional potentials. The transparency of the red dashed lines is proportional to the probability density function (PDF). Insets show the distribution for $A_{\rm s}$ as blue dash-dotted lines. Left and right panels correspond to CH$_i-$CH$_2-$CH$-$CH$_j$ and CH$_i-$CH$_2-$C$-$CH$_j$ torsions, respectively. Both $u^{\rm tors}/k_{\rm B}$ and $A_{\rm s}/k_{\rm B}$ are expressed in units of K.}
		\label{fig:dihedral_uncertainty}
	\end{figure}

	Figure \ref{fig:nonbonded_uncertainty} depicts the MCMC non-bonded parameters for CH$_3$, CH$_2$, CH, and C united-atom sites ($\epsilon_{\rm CH_3}$, $\sigma_{\rm CH_3}$, $\epsilon_{\rm CH_2}$, $\sigma_{\rm CH_2}$, $\epsilon_{\rm CH}$, $\sigma_{\rm CH}$, $\epsilon_{\rm C}$, and $\sigma_{\rm C}$) which are used for MCMC-nb and MCMC-nb-tors. Note that $\lambda_{\rm CH_3} = \lambda_{\rm CH_2} = \lambda_{\rm CH} = \lambda_{\rm C} = 16$. Parameters are assumed to be transferable, e.g., the CH$_2$ MCMC parameters are sampled independent of the CH$_3$ MCMC parameters. An important observation from Figure \ref{fig:nonbonded_uncertainty} is that the ranges of MCMC sampled CH and C non-bonded parameters are considerably larger (on a percent basis) than those for CH$_3$ and CH$_2$. In fact, the MCMC ranges for CH and C encompass both the ``long'' and ``generalized'' parameters, suggesting that these are actually indistinguishable parameter sets.
	
	The MCMC non-bonded parameters for CH$_3$ and CH$_2$ sites were determined in our previous study using a likelihood function based on experimental saturated liquid density and saturated vapor pressure data for ethane, propane, \textit{n}-butane, and \textit{n}-octane \cite{Postdoc_2}. By contrast, the MCMC parameters for CH and C sites are obtained from the scoring function reported by Mick et al. \cite{Potoff_branched} that depends on several vapor-liquid coexistence properties for a diverse set of branched alkanes. Details regarding the generation of MCMC parameter sets from the scoring function are found in Section \ref{SI:MCMC from scoring function} of Supplementary Material. Tabulated non-bonded MCMC parameter sets $(\epsilon$ and $\sigma)$ are provided in Section \ref{SI:Tabulated_MCMC} of Supplementary Material.
		
%	Translating the scoring function into a Bayesian context is achieved by modeling the $\epsilon$-$\sigma$ CH and C uncertainties with a multivariate normal distribution, where the covariance matrix was obtained by assuming that the ``generalized'' CH and C parameter set should not be distinguishable (at the 95~\% confidence level) from the ``long'' parameter set.
	
%    We apply the common assumption of transferability between UA sites, which implies that the parameter correlation between different UA sites, e.g., between $\sigma_{\rm CH_3}$ and $\sigma_{\rm CH}$, is assumed to be negligible. In other words, we only account for the parameter correlation between $\epsilon_{ii}$-$\sigma_{ii}$ sets of the same UA site. The reason for this assumption is the reduced complexity of performing four independent two-dimensional MCMC runs compared to one eight-dimensional MCMC run.
    
    \begin{figure}[H]
    	\centering
    	\includegraphics[width=6.4in]{MCMC_nonbonded.pdf}
    	\caption{Uncertainty in non-bonded parameters determined with Markov Chain Monte Carlo (MCMC). The MiPPE generalized and S/L parameters are also included as a reference \cite{Mie,Potoff_branched}. Top left, top right, bottom left, and bottom right panels correspond to CH$_3$, CH$_2$, CH, and C parameters, respectively. Note that a small discrepancy exists between the generalized C parameters in Table 1 of Reference \citenum{Potoff_branched} and the optimal region depicted in Figure 1 of Reference \citenum{Potoff_branched}. The generalized C parameters shown here were obtained from private communication of tabulated scoring function values and are, thus, slightly different than those reported by Mick et al.\cite{Potoff_branched}}		
    	%	\caption{Uncertainty in non-bonded potentials. Black line is the Potoff non-bonded potential. Red lines correspond to the 200 MCMC sampled parameter sets used in this study. Insets show the distribution for $\epsilon$ and $\sigma$.}
    	\label{fig:nonbonded_uncertainty}
    \end{figure}
    
%    Note that the uncertainties in the CH and C non-bonded parameters are considerably larger than those for CH$_3$ and CH$_2$. Note that the CH$_3$ parameters contribute to the majority of non-bonded interactions (both between different molecules as well as the 1-5 and 1-6 interactions within the same molecule). Therefore, the relatively small uncertainties assigned to the CH$_3$ parameters (recall Figure \ref{fig:nonbonded_uncertainty}) are likely the cause for the negligible impact of non-bonded uncertainties.
	
%    Two important assumptions are made in this UQ analysis. First, we not only apply the common assumption of transferability between UA sites, but we also assume that there is zero correlation between different UA sites \cite{Mess4}. Second, we assume that the uncertainties obtained from as the likelihood and scoring functions depend only on thermodynamic properties at vapor-liquid coexistence conditions, we assume that the  In other words, a variation be 	The CH$_3$ parameters for ethane are assumed to be transferable to longer \textit{n}-alkanes
%	
%%	  such that  m  reproduced  the average and covariance matrix were fit to the  
%%	
%%	such that the short/long
%%	
%%	The uncertainty in the CH and C parameter sites were obtained from Reference BLANK. Rather than apply 
%	
%    In summary, the following assumptions are made in this uncertainty analysis: 
%	
%	\begin{enumerate}
%		\item Correlation between different UA sites is zero, i.e., all UA sites are transferable
%		\item $\lambda = 16$, i.e.,  only $\epsilon$ and $\sigma$ have non-zero uncertainty
%		\item Likelihood function and scoring function depend only on thermodynamic properties at vapor-liquid coexistence conditions
%	\end{enumerate}

%		\item CH$_3$ parameters for ethane are transferable to \textit{n}-alkanes and branched alkanes
%\item CH$_2$ parameters for propane, \textit{n}-butane, and \textit{n}-octane are transferable to branched alkanes
%\item CH and C parameters for branched alkanes are transferable to challenge compound
	
	
	
%	With the assumption of sequential transferability from CH$_3$ 
%	
%	The uncertainty for the non-bonded Mie 16-6 parameter sets in $\epsilon_{CH_3}$ non-bonded Mie 16-6 parameters
%	
%	To simplify the uncertainty analysis, we assume the correlation between Mie parameters of different UA sites is zero. Therefore, we only account for the correlation between $\epsilon$ and $\sigma$ of a given UA site type.  
%	
%	The Potoff ``generalized'' CH and C parameter set is an attempt at a completely transferable set. However, since the ``generalized'' parameters performed poorly for some compounds, the S/L parameter set was proposed, where the ``short'' and ``long'' parameters are implemented when the number of carbons in the backbone is $\le 4$ and $> 4$, respectively.
	

	
	%     Of the 81 non-bonded interactions, 25 are CH$_3$-CH$_3$, 
	
	%     Since 45 ($5 \times 5 + 5 \times 4$) of the 81 $(9 \times 9)$ non-bonded interactions are affected by the CH$_3$ sites, the CH$_3$ uncertainties  
	
%	\begin{enumerate}
%		\item Potoff force field proved to be most reliable in previous study
%		\item United-atom, Mie 16-6
%		\item AUA4m considered modifying torsional barriers for CH$_2$-CH$_2$ by 15~\% and 40~\% for internal and terminal torsions, respectively.
%		\item Include uncertainty in $\epsilon$, $\sigma$, and $U^{\rm tors}$
%		\item Plots of MCMC samples and maybe the Mie potentials and torsional barriers explicitly
%	\end{enumerate}
	
	\subsection{Simulation set-up}
		
	Historically, non-equilibrium molecular dynamics (NEMD) has been preferred for highly viscous systems \cite{McCabe2001,Liu2015}. However, in our recent publication we successfully predicted the viscosity of 2,2,4-trimethylpentane at 293 K and 1000 MPa (the highest pressure required for the challenge) with equilibrium molecular dynamics (EMD) \cite{Postdoc_3}. Consistent with our previous study, we perform EMD simulations using GROMACS version 2018 with ``mixed'' (single and double) precision \cite{GROMACS_2018}. Example GROMACS input files (.top, .gro. and .mdp) with corresponding shell and Python scripts for preparing, running, and analyzing simulations are provided in Section \ref{SI:Gromacs input files} of Supplementary Material.
	
%	GROMACS is compiled using GNU 7.3.0, OpenMPI enabled, and GPU support disabled. The simulations are run using Linux 4.4.0-112-generic x86\_64 on an Intel(R) Xeon(R) CPU E5-2699 v4 @ 2.20GHz machine. 
	
	%In addition, the shell and python scripts used for preparing and analyzing simulations are available on GitHub \cite{BLANK}. 
	
    We utilize the same simulation specifications as our previous study \cite{Postdoc_3}. The general simulation specifications are provided in Table \ref{tab:sim_specs}. Good performance is achieved with a relatively long time-step of 2 fs because there are no fast vibrating modes. Specifically, all bond lengths are fixed and the UA model does not include explicit hydrogen sites. Bonds are constrained using LINear Constraint Solver (LINCS) with a LINCS-order of eight \cite{Hess1998,Hess2008}. 
       
    Note that we employ a 1.4 nm non-bonded cut-off with analytical tail corrections (where GROMACS only includes the contribution from the $r^{-6}$ term) instead of the 1.0 nm cut-off utilized to develop the MiPPE potential \cite{Mie,Potoff_branched}. Although a 1.0 nm cut-off is reliable for most compounds, it can lead to spurious viscosity estimates and even unstable simulations for large molecules, e.g., \textit{n}-hexadecane and \textit{n}-docosane \cite{Postdoc_3}. For this reason, we incur the additional computational cost with the longer cut-off so as to avoid any unforeseen simulation anomalies \cite{Leonard2018}. 
	
	Finite size effects for viscosity are typically negligible with a 200 or 400 molecule system \cite{Postdoc_3,Kim2018}. The MiPPE simulations utilize the larger system size only for $P \le 500$ MPa, while the smaller system size is favored at higher pressures due to the significant increase in simulation time (see Table \ref{tab:production times}). To reduce the computational cost, the MCMC-nb and MCMC-nb-tors simulations use 200 molecules at each pressure. Note that because viscosity is a collective property the uncertainty does not depend on the number of molecules.
	 
%	 For example, our previous study demonstrates that, with a 2 fs time-step, a cut-off distance of 1.4 nm or greater is required for \textit{n}-alkanes longer than \textit{n}-dodecane.
	 
	\begin{table}[htb!]
		\caption{General simulation specifications.} \label{tab:sim_specs}
		\begin{center}
			\begin{tabular}{|c|c|}
				\hline
				Time-step (fs) & 2 \\
				Cut-off length (nm) & 1.4 \\
				Tail-corrections & $U$ and $P$ \\
				Constrained bonds & LINCS \\
				LINCS-order & 8 \\			     
				Number of molecules & 200 or 400 \\
				\hline        
			\end{tabular}
		\end{center}
	\end{table}

	We perform a sequence of six simulation stages: energy minimization, $NPT$ equilibration, $NPT$ production, energy minimization, $NVT$ equilibration, and $NVT$ production. The average box size from the $NPT$ production stage is utilized in the second energy minimization and subsequent $NVT$ stages. Table \ref{tab:thermostats_barostats} lists the integrators, thermostats, barostats, and simulation time used for each $NPT$ and $NVT$ equilibration and production stage. These specifications are also the same as our previous study, with the exception of the $NVT$ production simulation times, which are state point dependent. The specific production times for the $NVT$ production stage are provided in Table \ref{tab:production times}. Note in Table \ref{tab:production times} that we include simulations at an additional pressure of 975 MPa to reduce the uncertainties in our high pressure viscosity estimates. 
	
	%in the empirical model fits at high pressures.

%	\begin{table}[htbp!]
%		\caption{Simulation specifications for equilibration (Equil.) and production (Prod.) stages.} \label{tab:thermostats_barostats}
%		%    	\begin{center}
%		\begin{tabular}{|p{2.6cm}|c|c|c|c|}
%			\hline
%			& $NPT$ Equil. & $NPT$ Prod. & $NVT$ Equil. & $NVT$ Prod. \\ \hline
%			Simulation time (ns) & 1 & 1 & 1 & 1 to 48 \\ \hline
%			Integrator & Velocity Verlet \cite{Swope1982}  & Leap frog \cite{Hockney1974} & Velocity Verlet & Velocity Verlet \\ \hline 
%			Thermostat & Velocity rescale \cite{Bussi2007} & Nos{\'e}-Hoover \cite{Hoover1985,Nose1984} & Nos{\'e}-Hoover & Nos{\'e}-Hoover \\ \hline 
%			Thermostat time-constant (ps) & 1.0 & 1.0 & 1.0 & 1.0 \\ \hline
%			Barostat & Berendsen \cite{Berendsen1984} & Parrinello-Rahman \cite{Nose1983,Parrinello1981} & N/A & N/A \\ \hline
%			Barostat time-constant (ps) & 1.0 & 5.0 & N/A & N/A \\ \hline
%			Barostat compressibility (1/bar) & 4.5E-5 & 4.5E-5 & N/A & N/A \\
%			\hline
%		\end{tabular}
%		%    	\end{center} 
%	\end{table}

	\begin{table}[htbp!]
		\caption{Simulation specifications for equilibration (Equil.) and production (Prod.) stages. $t_{\rm sim}$ is the simulation time, $\tau_{T}$ is the thermostat time-constant, $\tau_{P}$ is the barostat time-constant, and $\zeta_{P}$ is the barostat compressibility.} \label{tab:thermostats_barostats}
		%    	\begin{center}
		\begin{tabular}{|l|c|c|c|c|}
			\hline
			& $NPT$ Equil. & $NPT$ Prod. & $NVT$ Equil. & $NVT$ Prod. \\ \hline
			$t_{\rm sim}$ (ns) & 1 & 1 & 1 & 1 to 96 \\ \hline
			Integrator & Leap frog \cite{Hockney1974} & Leap frog & Velocity Verlet \cite{Swope1982} & Velocity Verlet \\ \hline 
			Thermostat & Velocity rescale \cite{Bussi2007} & Nos{\'e}-Hoover \cite{Hoover1985,Nose1984} & Nos{\'e}-Hoover & Nos{\'e}-Hoover \\ \hline 
			$\tau_{T}$ (ps) & 1.0 & 1.0 & 1.0 & 1.0 \\ \hline
			Barostat & Berendsen \cite{Berendsen1984} & Parrinello-Rahman \cite{Nose1983,Parrinello1981} & N/A & N/A \\ \hline
			$\tau_{P}$ (ps) & 1.0 & 5.0 & N/A & N/A \\ \hline
			$\zeta_{P}$ (1/bar) & 4.5 $\times 10^{-5}$ & 4.5 $\times 10^{-5}$ & N/A & N/A \\
			\hline
		\end{tabular}
		%    	\end{center} 
	\end{table}

	\begin{table}[htb!]
		\caption{State point specific production times. Pressure is prescribed only in $NPT$ equilibration and production stages.} \label{tab:production times}
		\begin{center}
			\begin{tabular}{|c|c|}
				\hline
				Pressure (MPa) & $NVT$ Prod. time (ns) \\ \hline
				0.1 & 1 \\
				25 & 1 \\
				50 & 1 \\
				100 & 1 \\			     
				150 & 1 \\
				250 & 2 \\
				400 & 4 \\
				500 & 8 \\
				600 & 8 \\			     
				700 & 16 \\
				800 & 64 \\
				900 & 72 \\
				975 & 96 \\
				1000 & 96 \\
				\hline        
			\end{tabular}
		\end{center}
	\end{table}
	
	A large number of replicate simulations are required at each state point to quantify and reduce the numerical uncertainty \cite{Maginn2018,Zhang2015}. We utilize between 40 and 80 independent replicates, where a greater number of replicates are needed for more viscous systems, i.e., at higher pressures. Each MiPPE replicate simulation utilizes the same force field parameters, whereas MCMC-nb and MCMC-nb-tors utilize different parameter sets for each replicate (see Figures \ref{fig:dihedral_uncertainty} and \ref{fig:nonbonded_uncertainty}). To ensure independence between replicates, the entire series of simulation stages is repeated, the energy minimization stages start with a different pseudo-random configuration, and the initial velocities are randomized for each $NPT$ and $NVT$ equilibration stage. The replicates, therefore, sample from a distribution of densities (i.e., box sizes) for a given pressure. 
	
	%each replicate having a different box size (i.e., density) for the $NVT$ stages.
	
	% . uch that each replicate uses a different box size for the $NVT$ stages., and each replicate uses a different box size for the $NVT$ stage. 
	
	\subsection{Data analysis}
	
	The post-simulation data analysis is identical to that prescribed in our previous study \cite{Postdoc_3}. In brief, we implement the Green-Kubo ``time-decomposition'' analysis \cite{Maginn2018,Zhang2015}
	\begin{equation} \label{eq:Green_Kubo}
	\eta(t) = \frac{V}{k_{\rm B} T N_{\rm reps}} \sum_{n=1}^{N_{\rm reps}} \int_{0}^{t}dt'\left\langle \tau_{\alpha\beta,n}(t') \tau_{\alpha\beta,n}(0)\right\rangle_{t_0,\alpha\beta}
	\end{equation} 
	where $t$ is time, $V$ is volume, $N_{\rm reps}$ is the number of independent replicate simulations, $\alpha$ and $\beta$ are $x, y, $ or $z$ Cartesian coordinates, $\tau_{\alpha\beta,n}$ is the $\alpha$-$\beta$ off-diagonal atomic stress tensor element for the $n^{\rm th}$ replicate, and $\langle \cdots \rangle_{t_0,\alpha\beta}$ denotes an average over twelve different time origins $(t_0)$ and all three unique off-diagonal $\tau_{\alpha\beta}$ components. 
	
	For an accurate integration of Equation \ref{eq:Green_Kubo}, $\tau_{\alpha\beta,n}$ is recorded every 6 fs (3 time-steps) for $P \le 700$ MPa. Less-frequent recording of $\tau_{\alpha\beta,n}$ is required to avoid unmanageable file sizes for the three highest pressures, as these systems require significantly longer simulations. Specifically, we record $\tau_{\alpha\beta,n}$ every 18 fs (9 time-steps) for $P = 800$ MPa, every 24 fs (12 time-steps) for $P = 900$ MPa, and every 36 fs (18 time-steps) for $P = 975$ and $1000$ MPa. We validate the results at high pressures with less-frequent recording of $\tau_{\alpha\beta,n}$ in Section \ref{SI:Output frequency} of Supplementary Material.  
	
%	Equation \ref{eq:Green_Kubo} also averages $N_{\rm reps}$ independent replicate simulations.
	
%	, Equation \ref{eq:Green_Kubo} averages the independent replicate simulations, twelve different time-origins, and all three unique off-diagonal stress tensor components.
	
	The force field viscosity is obtained by evaluating Equation \ref{eq:Green_Kubo} as $t \rightarrow \infty$, i.e., the infinite-time-limit viscosity $(\eta^\infty)$. As the long-time tail does not converge, we fit the ``running integral'' to a double-exponential function
	\begin{equation} \label{eq: Double exponential}
	\eta(t) = A \alpha \tau_1 \left(1-\exp{(-t/\tau_1)}\right) + A (1-\alpha) \tau_2 \left(1-\exp{(-t/\tau_2)}\right)
	\end{equation}
	where $A, \alpha, \tau_1, $ and $\tau_2$ are fitting parameters and $\eta^\infty = A \alpha \tau_1 + A (1-\alpha) \tau_2$. Our previous work provides details regarding the fitting procedure \cite{Postdoc_3}. 
	
%	 We fit Equation \ref{eq: Double exponential} using a weighted sum-squared-error objective function where the weights are equal to the inverse of the squared standard deviation $(\sigma^2_{\eta})$. The standard deviation with respect to time of the replicate simulations is fit to the model $A t^{b}$, where $A$ and $b$ are fitting parameters. To improve the fit of Equation \ref{eq: Double exponential}, we employ both a long-time and short-time cut-off. Specifically, only data for $t > 3$ ps are included in the fit, while data where $\eta(t) > 0.4 \times \eta^{\infty}$ are excluded \cite{Maginn2018,Zhang2015}. 
	 
	The uncertainty in $\eta$ is obtained by bootstrap re-sampling and reported at the 95~\% confidence level. Specifically, the fitting of Equation \ref{eq: Double exponential} is repeated thousands of times using randomly selected subsets of replicate simulations from which we obtain a distribution for $\eta^{\infty}$. Section \ref{SI:MCMC_analysis} of Supplementary Material validates the methodology for quantifying the MCMC-nb and MCMC-nb-tors uncertainties, namely, by bootstrap re-sampling replicate results obtained with different MCMC parameter sets. 
		
%	For the Potoff results, each replicate simulation utilizes the same force field parameters. By contrast, the MCMC-nb and MCMC-nb-tors replicates utilize different parameter sets. The 95~\% confidence interval is obtained from the distribution of bootstrap estimates for $\eta^\infty$. An example of the MCMC process is provided in Section \ref{SI:GK_analysis} of Supplementary Material.
%	
	
	% For the Potoff force field, the number of replicates in the subset is equal to the total number of replicates. The methodology is different for the MCMC simulations. In this case, two different rounds of re-sampling are performed. First, only 40 replicates are subsampled from the total of 200 replicates. This is done to randomize the MCMC parameter sets. Second, standard bootstrapping with replacement is performed from these 40 replicates. Both layers of this process are repeated hundreds of times. For both Potoff and MCMC force fields, a 95~\% confidence interval is obtained from the distribution of bootstrap estimates for $\eta^\infty$. An example of the MCMC process is provided in Section \ref{SI:GK_analysis} of Supplementary Material.
	
	\subsection{Pressure-viscosity coefficient} \label{Pressure-viscosity coefficient}
	
    The simulated $\eta$ values with respect to $P$ are fit to four different empirical models from which the pressure-viscosity coefficient $(\alpha)$ is calculated
    \begin{equation}
    \alpha \equiv \frac{d\ln(\eta)}{dP}
    \end{equation}
    As viscosity ranges over several orders of magnitude, the objective function for fitting is the sum-squared-error of log$_{10}(\eta)$. Only the MiPPE results are included in the fit.
    
    %, as these simulation results account for more sources of uncertainty than the Potoff and MCMC-nb results. 
    
    %The pressure-viscosity coefficient for each empirical fit is obtained by differentiation of $\ln(\eta)$ with respect to $P$.
           
    The first empirical model we implement is the traditional Barus expression \cite{Bair2016,Barus1893}
	\begin{equation} \label{eq:Barus}
	\eta = \eta_0 \exp(\alpha P)
	\end{equation}
	where the fitting parameters $\eta_0$ and $\alpha$ correspond to the zero-pressure viscosity and the pressure-viscosity coefficient, respectively. The second model is the popular Roelands equation \cite{Bair2016,Roelands1966}	
	\begin{equation} \label{eq:Roelands}
	\eta = \eta_p \left(\frac{\eta_0}{\eta_p}\right)^{\left(\frac{P_p - P}{P_p}\right)^Z}
	\end{equation}
	where $\eta_0$ and $Z$ are fitting parameters and $\eta_p = 6.31 \times 10^{-5}$ Pa-s and $P_p = -0.196$ GPa. The third model is an alternative form of Equation \ref{eq:Roelands} where $\eta_p$ and $P_p$ are additional fitting parameters (rather than fixed constants). This four parameter Roelands equation, which we refer to as ``Roelands-Modified,'' is more flexible than the standard Roelands model and can predict super-Arrhenius behavior. The fourth model is a hybrid McEwen-Paluch expression \cite{Bair2016}	
	\begin{equation} \label{eq:McEwen-Paluch}
	\eta = \eta_0 \left(1 + \frac{a_0}{q} P\right)^q \exp\left(\frac{C_{\rm F} P}{P_\infty - P}\right)
	\end{equation}
	where $\eta_0$, $a_0$, $q$, $C_{\rm F}$, and $P_\infty$ are fitting parameters.  Note that, although the Roelands-Modified equation can predict super-Arrhenius behavior, only the McEwen-Paluch model is capable of representing an inflection point, i.e., a transition from Arrhenius to super-Arrhenius behavior. By contrast, the Barus and Roelands models are only capable of fitting sub-Arrhenius and Arrhenius-like data.
	
	%Note that the Barus model is obtained from the hybrid McEwen-Paluch model when $a_0 = 0$, $q=1$, and $P_\infty = P + 1$. 
		
	\section{Results} \label{Results}    
	
	Table \ref{tab:tabulated_values} provides tabulated values for viscosity $(\eta)$ and density $(\rho)$ for MiPPE, MCMC-nb, and MCMC-nb-tors at the prescribed temperature $(T)$ and pressure $(P)$. The uncertainties (expressed at the 95~\% confidence level) for $T$, $\rho$, and $P$ are computed with the standard deviation of the replicate simulation averages, while those for $\eta$ are obtained from bootstrap re-sampling. Note that the average temperature slightly decreases with increasing pressure, but $T$ is always within the 0.3 K uncertainty stipulated by the IFPSC10 experimental measurements. Also, the average pressures tend to be lower than the prescribed pressure. For most state points, however, the combined uncertainty in pressure from experiment (the greater of 1 MPa and 0.4~\%) and simulation is sufficient to account for this discrepancy. 
	
	%Table \ref{tab:tabulated_values} also contains simulation results at an additional pressure of approximately 975 MPa. We include this state point to reduce the uncertainties in the empirical model fits at high pressures.
	
	\begin{table}[htb!]
		\caption{Simulation results for MiPPE, MCMC-nb, and MCMC-nb-tors. Subscripts and superscripts (when present) correspond to the lower and upper 95~\% confidence intervals, respectively. If only a subscript is present the lower and upper uncertainties are approximately equal (i.e., the bootstrap distribution is essentially symmetric) and only the average uncertainty is reported.} \label{tab:tabulated_values}
		\begin{center}
%			\begin{tabular}{|c|c|c|c|c|c|c|c|}
			\begin{tabular}{|c|c|c|c|c|c|}
				\hline
%				\multicolumn{1}{|c}{} & \multicolumn{1}{|c}{} & \multicolumn{2}{|c|}{Potoff} & \multicolumn{2}{|c|}{MCMC-nb} & \multicolumn{2}{|c|}{MCMC-nb-tors}  \\ \hline
%				$T$ (K) & $P$ (MPa) & $\rho$ (kg/m$^3$) & $\eta$ (Pa-s) & $\rho$ (kg/m$^3$) & $\eta$ (Pa-s) & $\rho$ (kg/m$^3$) & $\eta$ (Pa-s) \\ \hline 
%				$T$ (K) & 	$\rho$ (kg/m$^3$) & 	$P$ (MPa) & 	$\eta_{\rm Potoff}$ (mPa-s) & 	$\eta_{\rm MCMC-nb}$ (mPa-s) & 	$\eta_{\rm MCMC-nb-tors}$ (mPa-s) \\ \hline
				\multicolumn{1}{|c}{$T$ (K)} & \multicolumn{1}{|c}{$\rho$ (kg/m$^3$)} & \multicolumn{1}{|c}{$P$ (MPa)} & \multicolumn{3}{|c|}{$\eta$ (10$^{-3}$ Pa-s)} \\ \hline
				 &  &  & MiPPE & MCMC-nb & MCMC-nb-tors \\ \hline 
				292.854$_{0.095}$ & 	716.46$_{0.46}$ & 	-2.08$_{0.83}$ & 	0.605$_{0.045}$ & 	0.565$^{0.048}_{0.038}$ & 	0.578$^{0.059}_{0.043}$ \\
				292.851$_{0.085}$ & 	734.67$_{0.44}$ & 	22.9$_{1.1}$ & 	0.803$^{0.078}_{0.098}$ & 	0.786$^{0.079}_{0.062}$ & 	0.763$^{0.087}_{0.056}$ \\
				292.847$_{0.089}$ & 	749.22$_{0.45}$ & 	47.9$_{1.2}$ & 	0.957$^{0.076}_{0.061}$ & 	0.965$^{0.099}_{0.070}$ & 	1.014$_{0.091}$ \\
				292.85$_{0.10}$ & 	772.24$_{0.37}$ & 	98.3$_{1.3}$ & 	1.60$_{0.14}$ & 	1.50$_{0.11}$ & 	1.59$_{0.13}$ \\
				292.838$_{0.088}$ & 	790.25$_{0.35}$ & 	148.3$_{1.6}$ & 	2.18$^{0.24}_{0.19}$ & 	2.10$^{0.21}_{0.17}$ & 	2.18$^{0.28}_{0.21}$ \\
				292.819$_{0.096}$ & 	818.22$_{0.31}$ & 	248.5$_{2.0}$ & 	4.56$_{0.45}$ & 	4.25$_{0.34}$ & 	4.36$_{0.38}$ \\
				292.816$_{0.090}$ & 	849.17$_{0.38}$ & 	398.7$_{2.7}$ & 	13.5$_{1.2}$ & 	12.5$_{1.2}$ & 	14.1$_{1.2}$ \\
				292.790$_{0.099}$ & 	865.64$_{0.49}$ & 	498.4$_{4.6}$ & 	26.6$^{3.8}_{3.1}$ & 	28.5$_{2.7}$ & 	28.6$_{3.1}$ \\
				292.774$_{0.055}$ & 	879.9$_{1.2}$ & 	598.5$_{9.5}$ & 	65.0$_{9.6}$ & 	66.2$_{7.5}$ & 	61.8$^{6.8}_{4.8}$ \\
				292.763$_{0.041}$ & 	892.6$_{1.2}$ & 	698$_{10}$ & 	175$_{29}$ & 	-- & 	-- \\
				292.752$_{0.026}$ & 	903.5$_{1.1}$ & 	806$_{14}$ & 	523$_{68}$ & 	-- & 	-- \\
				292.738$_{0.022}$ & 	914.4$_{1.0}$ & 	899$_{11}$ & 	1480$_{206}$ & 	-- & 	-- \\
				292.725$_{0.020}$ & 	921.2$_{1.1}$ & 	973$_{16}$ & 	3528$_{372}$ & 	-- & 	-- \\
				292.719$_{0.013}$ & 	923.7$_{1.2}$ & 	1000$_{17}$ & 	4948$_{554}$ & 	-- & 	-- \\
				\hline
			\end{tabular}
		\end{center} 
	\end{table}
	
	%%% Prior to discussing Bair's data
%	The values from Table \ref{tab:tabulated_values} are also depicted in Figure \ref{fig:viscosity_pressure} along with the model fits to the MiPPE values and the available experimental viscosity at saturation pressure \cite{TDE}. An inflection point, suggesting super-Arrhenius behavior, is observed in the top panel of Figure \ref{fig:viscosity_pressure} around 200 MPa. The bottom panel of Figure \ref{fig:viscosity_pressure} presents the percent deviation between the McEwen-Paluch empirical model fit $(\eta_{\rm MP})$ and the simulation results, experimental data point, and the other three empirical model fits. This panel is useful for comparing the MiPPE, MCMC-nb, and MCMC-nb-tors uncertainties and for visualizing the goodness of fit with the various empirical models. For example, note that both the McEwen-Paluch and Roelands-Modified fits agree with nearly all of the simulation values to within their corresponding uncertainties.
	
	The values from Table \ref{tab:tabulated_values} are also depicted in Figure \ref{fig:viscosity_pressure} along with the model fits to the MiPPE values and the IFPSC10 experimental data (see Section \ref{SI:Exp data} of Supplementary Material). Recall that these experimental data were made available only after submitting our simulation results to IFPSC10. Prior to the challenge, the Thermodynamic Research Center database contained only a single experimental viscosity value (at saturation pressure) for this compound \cite{TDE}.
	
    The top panel of Figure \ref{fig:viscosity_pressure} demonstrates that an inflection point, suggesting super-Arrhenius behavior, is observed in the simulation results around 200 MPa. However, the inflection point for the IFPSC10 experimental data is much less pronounced, manifested by large systematic deviations between simulation and experimental values for $P \ge 700$ MPa. 
	
	The bottom panel of Figure \ref{fig:viscosity_pressure} presents the percent deviation between the McEwen-Paluch empirical model fit $(\eta_{\rm MP})$ and the simulation results, IFPSC10 experimental data, and the other three empirical model fits. This panel demonstrates that near-quantitative agreement is achieved between the simulation results and the IFPSC10 experimental data (to within the combined uncertainties) for pressures ranging from 0.1 MPa to 600 MPa. Furthermore, this panel is useful for comparing the MiPPE, MCMC-nb, and MCMC-nb-tors uncertainties and for visualizing the goodness of fit with the various empirical models. For example, note that both the McEwen-Paluch and Roelands-Modified fits agree with nearly all of the simulation values to within their corresponding uncertainties.  
	
	\clearpage
	\newpage
	
	\begin{figure}[H]
		\centering
		\includegraphics[width=3.2in]{viscosity_pressure_results.pdf}
		\caption{Pressure-viscosity results for MiPPE (red squares), MCMC-nb (green circles), and MCMC-nb-tors (blue diamonds). Top panel plots log$_{10}(\eta)$-$P$ where different line colors and styles represent different empirical model fits (Equations \ref{eq:Barus}, \ref{eq:Roelands}, and \ref{eq:McEwen-Paluch}) to MiPPE values. Bottom panel is a percent deviation plot relative to the McEwen-Paluch fit $(\eta_{\rm MP})$. Experimental viscosity data above saturation pressure were not available prior to the challenge. Error bars represent 95~\% confidence intervals obtained from bootstrap re-sampling. If not depicted, uncertainties are smaller than the symbol size.}
		%%% Prior to experimental data
		%Viscosity-pressure results for MiPPE (red squares), MCMC-nb (green circles), and MCMC-nb-tors (blue diamonds). Top panel plots log$_{10}(\eta)$-$P$ where different line colors and styles represent different empirical model fits (Equations \ref{eq:Barus}, \ref{eq:Roelands}, and \ref{eq:McEwen-Paluch}) to MiPPE values. Bottom panel is a percent deviation plot relative to the McEwen-Paluch fit $(\eta_{\rm MP})$. Experimental viscosity point at saturation pressure is included as a reference \cite{TDE}. Error bars represent 95~\% confidence intervals obtained from bootstrap re-sampling. Experimental uncertainty is smaller than the symbol size.} %%%
		\label{fig:viscosity_pressure}
	\end{figure}
	
	Recall that the MiPPE results only account for numerical uncertainties, while MCMC-nb accounts for numerical and non-bonded parameter uncertainties, and MCMC-nb-tors accounts for numerical, non-bonded and torsional parameter uncertainties. Notice in Table \ref{tab:tabulated_values} and Figure \ref{fig:viscosity_pressure} that the MiPPE, MCMC-nb, and MCMC-nb-tors uncertainties are approximately equal. This somewhat surprising result supports the conclusion that the non-bonded and torsional uncertainties are negligible compared to the numerical uncertainties in the Green-Kubo viscosity. 
	
%	Although somewhat counter intuitive, the reason why the MCMC-nb and MCMC-nb-tors uncertainties are smaller than the Potoff uncertainties is because the latter uses fewer replicate simulations. If the non-bonded and torsional uncertainties are negligible compared to the numerical uncertainties, the increase in $N_{\rm reps}$ reduces the numerical uncertainty and, consequently, the overall uncertainty decreases as well. Therefore, 
	
%	For 224TMH, the CH$_3$ parameters contribute to the majority of non-bonded interactions (both between different molecules as well as the 1-5 and 1-6 interactions within the same molecule). Therefore, the relatively small uncertainties assigned to the CH$_3$ parameters (recall Figure \ref{fig:nonbonded_uncertainty}) are likely the cause for the negligible impact of non-bonded uncertainties.
%	


%    Figure \ref{fig:uncertainties} compares the different sources of uncertainty. The Potoff results only account for numerical uncertainties while MCMC-nb also accounts for non-bonded parameter uncertainty and MCMC-nb-tors also accounts for torsional uncertainties. Note that the reason why the MCMC-nb and MCMC-nb-tors uncertainties are larger than the Potoff uncertainties is because the latter uses fewer replicate simulations. When the non-bonded and torsional uncertainties are negligible compared to the numerical uncertainties, this increase in $N_{\rm reps}$ reduces the numerical uncertainty and, consequently, the overall uncertainty decreases as well.
%        
%%     only 40 replicates are used for the latter while 200 replicates are used for the former. 
%
%	\begin{figure}[htb!]
%		\centering
%	%	\includegraphics[width=6.4in]{viscosity_pressure.pdf}
%		\caption{Uncertainty distributions for Potoff, MCMC-Mie, and MCMC-Mie-tors.}
%		\label{fig:uncertainties}
%	\end{figure}

    Figure \ref{fig:viscosity_pressure_coefficent} presents the predicted pressure-viscosity coefficient $(\alpha)$, as determined by fitting the MiPPE results to Equations \ref{eq:Barus}, \ref{eq:Roelands}, and \ref{eq:McEwen-Paluch}. The uncertainties in $\alpha$ are obtained with bootstrap re-sampling for the empirical model fits. 
    
	\begin{figure}[htb!]
		\centering
		\includegraphics[width=3.2in]{Pressure_viscosity_coefficient.pdf}
		\caption{Pressure-viscosity coefficient predicted with empirical model fits (Equations \ref{eq:Barus}, \ref{eq:Roelands}, and \ref{eq:McEwen-Paluch}). Dashed lines represent 95~\% confidence intervals obtained from bootstrap re-sampling.}
		\label{fig:viscosity_pressure_coefficent}
	\end{figure}

As expected, the traditional Barus $\alpha$ value is constant with respect to pressure. By contrast, the Roelands $\alpha$ value actually decreases with increasing pressure, while the Roelands-Modified $\alpha$ value increases with respect to pressure but without a change from negative to positive slope. Only the McEwen-Paluch $\alpha$ value shows the marked change in slope which corresponds to an inflection point in the log$_{10}(\eta)$-$P$ plot. 

Note that the $\alpha$ magnitudes for all empirical models are reasonable (i.e., similar in magnitude to other lubricants \cite{Mundy1996,McCabe2001,Liu2015}) over the entire range of pressures. In comparison with the IFPSC10 experimental value at 900 MPa ($\alpha$ = 7.52), however, the MiPPE simulation results overestimate the pressure-viscosity dependence at high pressures.

% Equations \ref{eq:Barus}, \ref{eq:Roelands}, and \ref{eq:McEwen-Paluch}.
	
%	Although the deviations from the hybrid McEwen-Paluch model fit to the simulation results are significantly lower than those of the Roelands and Barus models, this should be anticipated considering the McEwen-Paluch model has five fitting parameters while the Barus and Roelands models only have two. Note that the four parameter Roelands-Modified model also has lower deviations than the Roelands and Barus models.
	
	Although the hybrid McEwen-Paluch model clearly reproduces the simulation results with lower deviations than those of the Roelands and Barus models (see Figures \ref{fig:viscosity_pressure} and \ref{fig:cross_validation}), this should be anticipated considering the McEwen-Paluch model has five fitting parameters while the Barus and Roelands models only have two. Note that the four parameter Roelands-Modified model also has lower deviations than the Roelands and Barus models. Therefore, it is possible that the McEwen-Paluch model actually over fits our simulation results. 
	
	To assess this possibility, Figure \ref{fig:cross_validation} presents the cross-validation results for each model. Specifically, we implement a Monte Carlo cross-validation scheme where thousands of random sub-samples are selected for the training and testing set. Approximately 70~\% of the MiPPE simulation results (10 pressures) are included in the training set while 30~\% (4 pressures) are excluded for the testing set. The $\eta$ value at a given pressure is randomly sampled from the bootstrapped simulation uncertainties, such that the complete set of 14 $\eta$ values varies for each round of cross-validation.
	
	% Specifically, according to the $\eta$ bootstrap uncertainties. 
	
	\begin{figure}[htb!]
		\centering
		\includegraphics[width=6.4in]{cross_validation.pdf}
		\caption{Monte Carlo cross-validation for empirical model fits (Equations \ref{eq:Barus}, \ref{eq:Roelands}, and \ref{eq:McEwen-Paluch}). MSE$_{\rm train}$ and MSE$_{\rm test}$ are the mean-squared-error for the ``training'' and ``testing'' sets, respectively. Left panel compares MSE$_{\rm train}$ and MSE$_{\rm test}$ for each model, while the right panel compares the McEwen-Paluch MSE$_{\rm test}$ with the MSE$_{\rm test}$ for the other three models.}
		\label{fig:cross_validation}
	\end{figure}
	
	The left panel of Figure \ref{fig:cross_validation} demonstrates that the mean-squared-error (MSE) for the training set is approximately equal to the MSE for the testing set of each model, suggesting that none of the models over fit the data. Note that, although somewhat counter intuitive, MSE$_{\rm test} < $ MSE$_{\rm train}$ typically denotes that the training set is ``easier'' to fit than the testing set. The right panel shows that only the Roelands-Modified model has a similar MSE to that of the McEwen-Paluch model for the same testing set. As both the Roelands-Modified and McEwen-Paluch models predict super-Arrhenius behavior, Figure \ref{fig:cross_validation} provides strong statistical evidence that the MiPPE force field predicts super-Arrhenius behavior. However, as the Roelands-Modified model does not predict that the slope of $\alpha$ with respect to $P$ changes signs, there exists some doubt whether an inflection point precedes the super-Arrhenius region.
	
	%	an inflection point followed by super-Arrhenius behavior 
	
	%	To quantitatively verify that the Potoff-MCMC force field predicts super-Arrhenius behavior at high pressures, we perform cross-validation between Equations BLANK and BLANK. 
	%	
	
	Table \ref{tab:smoothed_values} is included to facilitate scoring our entry for the 10$^{\rm th}$ Industrial Fluid Properties Simulation Challenge. Table \ref{tab:smoothed_values} provides ``smoothed'' $\eta$ and $\alpha$ values calculated with the McEwen-Paluch fit $(\eta_{\rm MP})$ to the MiPPE simulation results. As our initial fit under estimated the MiPPE $\eta$ value at atmospheric pressure, the values in Table \ref{tab:smoothed_values} are obtained by constraining $\eta_0$ in Equation \ref{eq:McEwen-Paluch} to reproduce the simulated viscosity at $P = 0.1$ MPa. Note that, despite performing simulations at both 975 and 1000 MPa, the uncertainty in $\eta$ at 1000 MPa is considerably larger than at the other pressures. 
	
	%Recall that simulation results at 975 MPa  As the simulation pressure is nearly 30 MPa lower than the prescribed 1000 MPa (see Table \ref{tab:tabulated_values}), the increased uncertainty in the McEwen-Paluch fit is caused by this extrapolation to 1000 MPa.
	
	The uncertainties in Table \ref{tab:smoothed_values} reflect both the simulation and experimental pressure uncertainties. Specifically, the McEwen-Paluch model is fit thousands of times using bootstrap re-sampling. Each fit is subsequently evaluated at $P^{\rm exp} \pm u_P^{\rm exp}$, where $u_P^{\rm exp}$ is the reported uncertainty in the experimental pressure $(P^{\rm exp})$. The lower and upper 95~\% confidence intervals are obtained from the distributions of $\eta_{\rm MP}$ computed at $P^{\rm exp} - u_P^{\rm exp}$ and $P^{\rm exp} + u_P^{\rm exp}$, respectively.
	
	% is obtained from the distribution of 
	
	%  over a range of  different values of $P$ that are sampled from a normal distribution with the standard deviation equal to $. 
	
	\begin{table}[htb!]
		\caption{Recommended (smoothed) values for the purpose of scoring our entry to the 10$^{\rm th}$ Industrial Fluid Properties Simulation Challenge. Uncertainties are expressed at the 95~\% confidence level. Pressure uncertainties are those reported for the experimental measurements.} \label{tab:smoothed_values}
		\begin{center}
			\begin{tabular}{|c|c|c|}
				\hline
				$P$ (MPa) & 	$\eta$ (10$^{-3}$ Pa-s) & 	$\alpha$ (1/GPa) \\ \hline
				0.1$_{1.0}$ & 	0.605$_{0.045}$ & 	13.1$_{1.8}$ \\
				25.0$_{1.0}$ & 	0.801$_{0.018}$ & 	10.168$_{0.091}$ \\
				50.0$_{1.0}$ & 	1.014$_{0.018}$ & 	8.87$_{0.26}$ \\
				100.0$_{1.0}$ & 	1.528$_{0.028}$ & 	7.74$_{0.23}$ \\
				150.0$_{1.0}$ & 	2.222$_{0.055}$ & 	7.31$_{0.11}$ \\
				250.0$_{1.0}$ & 	4.56$_{0.12}$ & 	7.16$_{0.11}$ \\
				400.0$_{1.6}$ & 	13.68$_{0.39}$ & 	7.59$_{0.21}$ \\
				500.0$_{2.0}$ & 	29.9$_{1.5}$ & 	8.10$_{0.22}$ \\
				600.0$_{2.4}$ & 	69.5$_{4.8}$ & 	8.77$_{0.18}$ \\
				700.0$_{2.8}$ & 	174$_{15}$ & 	9.58$_{0.20}$ \\
				800.0$_{3.2}$ & 	475$_{44}$ & 	10.57$_{0.41}$ \\
				900.0$_{3.6}$ & 	1449$_{177}$ & 	11.77$_{0.78}$ \\
				1000.0$_{4.0}$ & 	5040$_{863}$ & 	13.2$_{1.1}$ \\
				
% 				Old values
%				$P$ (MPa) & 	$\eta$ (10$^{-3}$ Pa-s) & 	$\alpha$ (1/GPa) \\ \hline
%				0.1$_{1.0}$ & 	0.605$_{0.045}$ & 	12.7$_{1.4}$ \\
%				25.0$_{1.0}$ & 	0.799$_{0.017}$ & 	10.165$_{0.079}$ \\
%				50.0$_{1.0}$ & 	1.012$_{0.018}$ & 	8.92$_{0.22}$ \\
%				100.0$_{1.0}$ & 	1.530$_{0.027}$ & 	7.79$_{0.20}$ \\
%				150.0$_{1.0}$ & 	2.229$_{0.051}$ & 	7.33$_{0.11}$ \\
%				250.0$_{1.0}$ & 	4.57$_{0.11}$ & 	7.14$_{0.10}$ \\
%				400.0$_{1.6}$ & 	13.65$_{0.42}$ & 	7.53$_{0.19}$ \\
%				500.0$_{2.0}$ & 	29.7$_{1.4}$ & 	8.02$_{0.19}$ \\
%				600.0$_{2.4}$ & 	68.2$_{4.4}$ & 	8.66$_{0.17}$ \\
%				700.0$_{2.8}$ & 	169$_{13}$ & 	9.46$_{0.22}$ \\
%				800.0$_{3.2}$ & 	455$_{42}$ & 	10.43$_{0.43}$ \\
%				900.0$_{3.6}$ & 	1367$_{173}$ & 	11.60$_{0.79}$ \\
%				1000.0$_{4.0}$ & 	4667$_{888}$ & 	13.0$_{1.2}$ \\
				
%				Old values
%				0.1$_{1.0}$ & 	0.605$_{0.045}$ & 	13.66$_{0.32}$ \\
%				25.0$_{1.0}$ & 	0.790$_{0.011}$ & 	10.380$_{0.055}$ \\
%				50.0$_{1.0}$ & 	1.004$_{0.012}$ & 	8.964$_{0.088}$ \\
%				100.0$_{1.0}$ & 	1.517$_{0.020}$ & 	7.769$_{0.072}$ \\
%				150.0$_{1.0}$ & 	2.209$_{0.033}$ & 	7.326$_{0.036}$ \\
%				250.0$_{1.0}$ & 	4.539$_{0.069}$ & 	7.174$_{0.047}$ \\
%				400.0$_{1.6}$ & 	13.68$_{0.28}$ & 	7.623$_{0.073}$ \\
%				500.0$_{2.0}$ & 	30.08$_{0.86}$ & 	8.160$_{0.074}$ \\
%				600.0$_{2.4}$ & 	70.3$_{2.6}$ & 	8.851$_{0.070}$ \\
%				700.0$_{2.8}$ & 	177.6$_{8.0}$ & 	9.70$_{0.10}$ \\
%				800.0$_{3.2}$ & 	493$_{27}$ & 	10.74$_{0.18}$ \\
%				900.0$_{3.6}$ & 	1531$_{114}$ & 	11.99$_{0.31}$ \\
%				1000.0$_{4.0}$ & 	5466$_{587}$ & 	13.51$_{0.49}$ \\
				\hline
			\end{tabular}
		\end{center} 
	\end{table} 
	
    \section{Discussion} \label{Discussion/Limitations}   

    Although the MiPPE viscosities for 224TMH are quantitatively reliable for $P \le 600$ MPa, the percent deviation relative to experiment is approximately 300\% at 1000 MPa. The rapid increase in $\eta$ with respect to $P$ also results in over prediction of the pressure-viscosity coefficient $(\alpha)$ at 900 MPa (11.77 compared to the IFPSC10 experimental value of 7.52). We suspect that this poor performance at the highest pressures is caused by the Mie 16-6 potential being overly repulsive at short distances \cite{Postdoc_3,Postdoc_2}. For this reason, MiPPE is more likely to predict super-Arrhenius behavior than traditional Lennard-Jones 12-6 force fields. Note that, as demonstrated in our previous study \cite{Postdoc_2}, we also expect the densities reported in Table \ref{tab:tabulated_values} at high pressures to be significantly lower than experimental densities.
    
    It is surprising that both the non-bonded and torsional parameter uncertainties are negligible compared to the numerical uncertainties in $\eta$ (see Figure \ref{fig:viscosity_pressure}). A possible explanation for why the non-bonded parameter uncertainty has a negligible impact on $\eta$ is that the CH$_3$ uncertainties are considerably smaller than those for CH$_2$, CH, and C (see Figure \ref{fig:nonbonded_uncertainty}). As 224TMH consists primarily of CH$_3$ sites, the larger uncertainties in CH$_2$, CH, and C appear to not affect the results significantly. 
    
%    For 224TMH, the CH$_3$ parameters contribute to the majority of non-bonded interactions (both between different molecules as well as the 1-5 and 1-6 interactions within the same molecule). Therefore, the relatively small uncertainties assigned to the CH$_3$ parameters (recall Figure \ref{fig:nonbonded_uncertainty}) are likely the cause for the negligible impact of non-bonded uncertainties.
    
    By contrast, no clear explanation exists for why the torsional parameter uncertainties do not affect the overall uncertainty in $\eta$. Nieto-Draghi et al. suggest that a 15~\% to 40~\% increase in the torsional barriers should increase the viscosities appreciably for similar compounds \cite{Nieto2006,Nieto2008}. However, we do not observe such a strong dependence. Specifically, the required increase in the torsional barriers is between 80~\% and 100~\% to achieve an increase of approximately 10~\% in viscosity (see Section \ref{SI:MCMC torsions} of Supplementary Material).
    
    % We propose that the discrepancy between our findings and those of Nieto-Draghi et al. arises as a consequence of differences between the force field forms
    
    % the united-atom sites and Mie 16-6 potential employed in this work and the anisotropic-united-atom sites and Lennard-Jones 12-6 potential employed by Nieto-Draghi et al.
    
    %arises either from using united-atom sites or the Mie 16-6 potential, whereas AUA4 utilizes anisotropic-united-atom sites with the Lennard-Jones 12-6 potential.
    
    
    
    % or the Mie 16-6 po the  This suggests that either this particular molecule or the Mie 16-6 potential is much less sensitive to the torsional parameters than the long straight-chained alkanes studied in Reference \cite{Nieto2006}.
        
%    Although the MiPPE force field demonstrates super-Arrhenius behavior, we should caution that this could be unique to the force field. Since the Mie 16-6 potential is known to be overly repulsive at short distances \cite{Postdoc_3,Postdoc_2}, it is possible that this causes the rapid increase in $\eta$ at high pressures. For example, as observed in our previous study \cite{Postdoc_2}, we expect the densities reported in Table \ref{tab:tabulated_values} at high pressures to be significantly lower than the experimental values.
    
    % by approximately $+$30~\% to $+$60~\%, with the deviations increasing with respect to pressure.
    
    %strongly from the experimental values.
    
    %This has proven to be the case for the viscosity-density trend, although we have assumed that the viscosity-pressure trend is reliable due to the simultaneous over prediction of both viscosity and pressure at high densities.
	
	Other studies \cite{Liu2015} correct for systematic errors in viscosity by normalizing $\eta$ with respect to an experimental viscosity value at low pressure. This approach would be possible for the challenge compound since a single experimental data point is available at saturation pressure. Although this may provide a more accurate prediction, we prefer not to use an empirical correction, especially from a single data point. Our goal, rather, is to truly test the predictive capabilities of the MiPPE force field.
	
%	The slow system dynamics (i.e., long rotational relaxation times) at high pressures require extremely long simulations and a large number of replicates. An attractive alternative is the so-called time-temperature superposition method, where simulations are performed at higher temperatures (to enhance the configurational sampling) and the viscosity at 293 K is obtained through extrapolation \citenum{Liu2015}. Despite some obvious benefits, we are wary of the inordinately large uncertainties that this method can produce (see Figure 11 of Ref. \citenum{Liu2015}). Determining the existence of super-Arrhenius behavior necessitates manageable uncertainties at high pressures. For this reason, we choose the more arduous brute-force approach.
	
	The slow system dynamics (i.e., long rotational relaxation times) at high pressures require extremely long simulations. Two attractive alternatives exist to enhance the configuration sampling by performing simulations at higher temperatures. The viscosity at the desired temperature is then obtained indirectly. 
	
	For example, the so-called time-temperature superposition method (TTS) combines the shear rate dependent viscosity at different (higher) temperatures into a single master curve for estimating viscosity at any temperature \cite{Khabaz2018}. Unfortunately, TTS is only applicable to NEMD simulations. An even simpler method, which is compatible with both NEMD and EMD, is to fit high-temperature viscosity values with an empirical model, e.g., the Vogel-Fulcher-Tammann-Hesse (VFTH) equation, and to then extrapolate to the desired temperature \cite{Liu2015,Khabaz2018}. Despite some obvious benefits, we are wary of the inordinately large uncertainties that this method can produce (see Figure 11 of Reference \citenum{Liu2015}). Determining the existence of super-Arrhenius behavior necessitates manageable uncertainties at high pressures. Another clear disadvantage of both methods is the need to simulate at multiple temperatures, while IFPSC10 only requires viscosity estimates at 293 K. For these reasons, we choose the more direct brute-force approach.	
	
%	Although time-temperature superposition is an attractive alternative, we are weary of the inordinately large uncertainties that this method can produce. If there is any hope to determine the existence of super-Arrhenius behavior, the uncertainties at high pressures must be manageable. 
%	
%	For this reason, we did not perform simulations at higher temperatures 
	
	\section{Conclusions} \label{Conclusions}
	
    Previous work demonstrated that the MiPPE force field provides reliable viscosities (typically within 10~\%) for well-studied \textit{n}-alkane and branched alkanes both at saturation and elevated pressures. For this reason, the MiPPE force field was chosen to predict the pressure-viscosity relationship of 2,2,4-trimethylhexane as part of the 10$^{\rm th}$ Industrial Fluid Properties Simulation Challenge. In addition, we investigate the force field parameter uncertainty in the simulation results with Bayesian inference. Specifically, the non-bonded and torsional parameters are varied from run to run according to Markov Chain Monte Carlo. Surprisingly, the non-bonded and torsional parameter uncertainties are typically negligible compared to the numerical uncertainties, i.e., the fluctuations in simulation output and post-processing. Furthermore, we use cross-validation model selection to verify the existence of faster-than-exponential, a.k.a. super-Arrhenius, behavior at high pressures. 
    
    Near-quantitative agreement with experimental viscosity data (to within the combined uncertainties) is observed from atmospheric pressure to 600 MPa. At the highest pressures, however, the MiPPE force field dramatically over predicts the viscosity and the pressure-viscosity coefficient. This suggests that the overly-repulsive Mie 16-6 non-bonded potential limits the pressure range over which the MiPPE force field is reliable.
	
	\section*{Supplementary Material}
	
    Section \ref{SI:Exp data} reports the IFPSC10 experimental data. Section \ref{SI:Gromacs input files} provides GROMACS input files. Section \ref{SI:MCMC from scoring function} describes how the CH and C non-bonded MCMC parameter sets are obtained. Section \ref{SI:MCMC_analysis} validates the MCMC-nb uncertainty quantification approach. Section \ref{SI:MCMC torsions} develops the $A_{\rm s}$ distribution used for the MCMC-nb-tors torsional parameters. Section \ref{SI:Tabulated_MCMC} provides tabulated values of the MCMC parameter sets. Section \ref{SI:Running integrals} presents the average Green-Kubo integrals for each state point. Section \ref{SI:Output frequency} investigates the impact of the output frequency at high pressures.
	
	\section*{Acknowledgments}
	
	We would like to acknowledge Jeffrey J. Potoff and Mohammad Soroush Barhaghi for their invaluable insights regarding the MiPPE force field. We are grateful for the internal review provided by Alta Y. Fang, Ian H. Bell, Andrei F. Kazakov, Daniel G. Friend, and Marcia L. Huber of the National Institute of Standards and Technology (NIST). We also appreciate the useful discussions and recommendations from Edward J. Maginn of The University of Notre Dame, Michael R. Shirts of The University of Colorado, J. Ilja Siepmann of The University of Minnesota, Jarod M. Younker of ExxonMobil, and John D. Chodera of the Memorial Sloan Kettering Cancer Center. 
	
	This research was performed while Richard A. Messerly held a National Research Council (NRC) Postdoctoral Research Associateship at NIST and while Michelle C. Anderson held a Summer Undergraduate Research Fellowship (SURF) position at NIST. 

	Commercial equipment, instruments, or materials are identified only in order to adequately specify certain procedures. In no case does such identification imply recommendation or endorsement by NIST, nor does it imply that the products identified are necessarily the best available for the intended purpose.
	
	Partial contribution of NIST, an agency of the United States government; not subject to copyright in the United States.
	
	\section*{References}
	
	\bibliographystyle{unsrt}
	\bibliography{IFPSC_10_references}
		
\end{document}
